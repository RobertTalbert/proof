\documentclass[11pt]{article}

\pagestyle{empty}                       %no page numbers
\thispagestyle{empty}                   %removes first page number
\setlength{\parindent}{0in}               %no paragraph indents

\usepackage{fullpage}
\usepackage[tmargin = 0.5in, bmargin = 1in, hmargin = 1in]{geometry}     %1-inch margins
\geometry{letterpaper}                  
\usepackage{graphicx}
\usepackage{amssymb}

% Default packages
\usepackage{latexsym}
\usepackage{amsfonts}
\usepackage{amsmath}
\usepackage{amsthm}
\usepackage{palatino}
\usepackage{hyperref}
\usepackage{multicol}
\usepackage{fancyhdr}
\usepackage{enumitem}
\usepackage{mathtools}

\def\pageturn{\vfill
\begin{flushright}
	\begin{small}
		Continued $\rightarrow$
	\end{small}
\end{flushright}
\newpage}

\newtheorem{proposition}{Proposition}
\newtheorem{lemma}{Lemma}


\def\ra{\rightarrow}

\begin{document}
	
	\thispagestyle{empty}
	\renewcommand{\headrulewidth}{0.0pt}
	\thispagestyle{fancy}
	\lhead{Prof. Talbert}
	\chead{MTH 210: Communicating in Mathematics}
	\rhead{November 16/19, 2012}
	\lfoot{}
	\cfoot{}
	\rfoot{}	
	
	\vspace*{0in}

		\begin{center}
			\begin{large}
			\textbf{Class Work: Inverse functions} \\
			\end{large}
			This is a full-time activity worth 10 points. 
			
		\end{center}
		

\section*{Problems of the Day}

\begin{enumerate}
	
	\item Let $A$ and $B$ be nonempty sets and let $f: A \to B$ be a bijection. Prove that for every $y \in B$, $(f \circ f^{-1})(b) = b$. 
	
	\item Let $f: \mathbb{R} \to \mathbb{R}$ be defined by $f(x) = e^x$. 
	\begin{enumerate}
		\item Explain why the inverse of this function is not a function. 
		\item Change the codomain of $f$ so that the inverse $f^{-1}$ will now be a function. What is the new codomain and why is $f^{-1}$ now a function? 
		\item Is $f^{-1}$ injective? Surjective? (Hint: You should not need to prove anything in this problem. Look back over your known theorem results.) 
		\item What's the more common name for $f^{-1}$, the inverse function for $f(x) = e^x$? What are its domain and codomain? 
		\item Explain how problem 1 in this activity proves the identities: 
		\[ e^{\ln x} = x,  \ \forall x > 0 \qquad \text{and} \qquad \ln(e^x) = x,  \ \forall x \in \mathbb{R} \]
	\end{enumerate} 

	
\end{enumerate}

\section*{Parameters}

If your group finishes your work, please hand it in at the end of class. If all groups finish by the end of class, we will take time to debrief the solutions to one or more of these. 

\end{document}