\documentclass[11pt]{article}

\pagestyle{empty}                       %no page numbers
\thispagestyle{empty}                   %removes first page number
\setlength{\parindent}{0in}               %no paragraph indents

\usepackage{fullpage}
\usepackage[tmargin = 0.5in, bmargin = 1in, hmargin = 1in]{geometry}     %1-inch margins
\geometry{letterpaper}                  
\usepackage{graphicx}
\usepackage{amssymb}

% Default packages
\usepackage{latexsym}
\usepackage{amsfonts}
\usepackage{amsmath}
\usepackage{amsthm}
\usepackage{palatino}
\usepackage{hyperref}
\usepackage{multicol}
\usepackage{fancyhdr}
\usepackage{enumitem}

\def\pageturn{\vfill
\begin{flushright}
	\begin{small}
		Continued $\rightarrow$
	\end{small}
\end{flushright}
\newpage}

\newtheorem{proposition}{Proposition}
\newtheorem{lemma}{Lemma}


\def\ra{\rightarrow}

\begin{document}
	
	\thispagestyle{empty}
	\renewcommand{\headrulewidth}{0.0pt}
	\thispagestyle{fancy}
	\lhead{Prof. Talbert}
	\chead{MTH 210: Communicating in Mathematics}
	\rhead{October 3, 2012}
	\lfoot{}
	\cfoot{}
	\rfoot{}	
	
	\vspace*{0in}

		\begin{center}
			\begin{large}
			\textbf{Class Work: Using the Division Algorithm in a Proof} \\
			\end{large}
			This is a FULL-TIME activity worth 10 points. 
			
		\end{center}
		

\section*{Problem of the Day}

Today you'll prove the proposition: 
\begin{proposition}
	The number $\sqrt{3}$ is irrational. 
\end{proposition}
You'll be proving this proposition in two stages, the first of which draws upon what you've learned about the Division Algorithm and its use in setting up cases. 


\begin{enumerate}
	\item Before proving Proposition 1, let's consider the following lemma\footnote{A ``lemma'' is a mini-theorem whose main purpose is to provide justification for a step in a larger proof.}: 
	
	\begin{lemma}
		For every integer $a$, if $a^2$ is divisible by $3$, then $a$ is divisible by $3$. 
	\end{lemma}
	
	\begin{enumerate}
		\item What does the contrapositive of Lemma 1 say? 
			\vspace{0.5in}

		\item Let's try to prove Lemma 1 by proving its contrapositive. What are you going to assume, and what are you going to prove? 
		\begin{itemize}
			\item We will assume: 
			\item We will try to prove: 
		\end{itemize}
		
		\textbf{Remember your eventual proof of this Lemma must not treat the statement to be proven as a fact. Don't assume what you are trying to prove is true. }
		
		\item What are the possible remainders you could get if you take $a$ and divide it by $3$? 
		
			\vspace{0.5in}

		\item Which, if any, of those remainders are ruled out in the contrapositive of Lemma 1? 
		
		\vspace{0.5in}
		
		\item Set up two cases for proving Lemma 1. 
		
		\begin{itemize}
			\item \textbf{Case 1:} 
			\item \textbf{Case 2:} 
		\end{itemize}
		
		\item On the very last page, there is space for a formal proof of Lemma 1. Sketch out your proof on a separate page, and then designate one person to write the proof up formally on the last page. 
	\end{enumerate}
	
	\pageturn
	
	\item Now that we've proven Lemma 1, we're ready to try a proof of the main Proposition. 
	\begin{enumerate}
		\item What overall strategy (direct proof, contraposition, contradiction, etc.) are you going to use? 
		
		\vspace{0.5in}
		
		\item What are you going to assume, and what are you going to try to show? 
		
		\vspace{1in}
		
		\item Use the space below or a blank page to sketch out your proof. The proof here might be very similar to the proof that $\sqrt{2}$ is irrational but you will need to use Lemma 1. Think about how Lemma 1 will play a role. Then write your formal proof in the final page. 
	\end{enumerate}
	
\end{enumerate}


\vfill

\section*{Extra Problem}

If your group completes the Problem of the Day successfully with time remaining, try this: 

\begin{proposition}
	The number $\sqrt{5}$ is irrational. 
\end{proposition}

What's a lemma you might need for this proof? Make sure you prove that as part of your work. \\

Can this method of proof be extended to prove that $\sqrt{6}$ is irrational? Why or why not? \\


\emph{Successful completion of each additional problem will add 1 point to both your Class Work score and the overall Class Work total. However, you must attain a score of at least 8/10 on the main Class Work to receive these points. (That is, you've completed all the problems and they are mostly correct.) }	

\section*{Futher practice}
For more practice, please try the following exercises from \S3.5, pages 153---157: 
\begin{itemize}
	\item 6, 7, 11, 12, 17, 18. 
\end{itemize}
\textbf{These are all optional. }
But if you submit writeups of one or more of these that represent good-faith efforts to complete the problem correctly, I will give you feedback on whatever you submit. 

\newpage

\begin{center}
	\begin{Large}
		Writeup for Class Work on \S3.5 (October 2)
	\end{Large}
	
\end{center}

\noindent
\textbf{Lemma 1:} For all integers $a$, if $3$ divides $a^2$, then $3$ divides $a$. \\

\noindent
\emph{Proof}: 

\vspace{5in}


\noindent
\textbf{Proposition 1:} The number $\sqrt{3}$ is irrational. \\

\noindent
\emph{Proof}:

\end{document}