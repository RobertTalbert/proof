\documentclass[11pt]{article}

\pagestyle{empty}                       %no page numbers
\thispagestyle{empty}                   %removes first page number
\setlength{\parindent}{0in}               %no paragraph indents

\usepackage{fullpage}
\usepackage[tmargin = 0.5in, bmargin = 1in, hmargin = 1in]{geometry}     %1-inch margins
\geometry{letterpaper}                  
\usepackage{graphicx}
\usepackage{amssymb}

% Default packages
\usepackage{latexsym}
\usepackage{amsfonts}
\usepackage{amsmath}
\usepackage{amsthm}
\usepackage{palatino}
\usepackage{hyperref}
\usepackage{multicol}
\usepackage{fancyhdr}
\usepackage{enumitem}

\def\pageturn{\vfill
\begin{flushright}
	\begin{small}
		Continued $\rightarrow$
	\end{small}
\end{flushright}
\newpage}

\newtheorem{proposition}{Proposition}
\newtheorem{lemma}{Lemma}


\def\ra{\rightarrow}

\begin{document}
	
	\thispagestyle{empty}
	\renewcommand{\headrulewidth}{0.0pt}
	\thispagestyle{fancy}
	\lhead{Prof. Talbert}
	\chead{MTH 210: Communicating in Mathematics}
	\rhead{October 5, 2012}
	\lfoot{}
	\cfoot{}
	\rfoot{}	
	
	\vspace*{0in}

		\begin{center}
			\begin{large}
			\textbf{Class Work: Mathematical Induction, day 2} \\
			\end{large}
			
		\end{center}
		

\section*{Problems of the Day}

\begin{enumerate}
	\item Write a complete proof of one of the five propositions you worked with on Friday (your choice). If you finished this on Friday, just make sure you have a clean writeup to turn in today. 
	\begin{description}
		\item[Proposition 1:] For each natural number $n$, 
		\[ 2 + 5 + 8 + \cdots + (3n-1) = \frac{n(3n+1)}{2} \]
		\item[Proposition 2:] For each natural number $n$, $4^n \equiv 1 \pmod 3$. 
		\item[Proposition 3:] For each natural number $n$, $x-y$ divides $x^n - y^n$. 
		\item[Proposition 4:] For each natural number $n$, $3$ divides $n^3 + 23n$.
		\item[Proposition 5:] Suppose we draw $n$ straight lines in the plane (which may cross but not lie coincident with each other), dividing it into a number of regions. Then for every natural number $n$, we may color each region either red or blue in such a way that no two neighboring regions have the same color. 
	\end{description}
	
	
	\item Below are some problems to consider that involve creating a conjecture and then proving it using mathematical induction. Choose one to do in your groups. 
	\begin{enumerate}
		\item Calculate the values of
		\[ \frac{1}{2} + \frac{1}{4} \qquad \frac{1}{2} + \frac{1}{4} + \frac{1}{8} \qquad \frac{1}{2} + \frac{1}{4} + \frac{1}{8}+ \frac{1}{16}\]
		Based on your results, make a conjecture about the value of the sum 
		\[ \frac{1}{2} + \frac{1}{4} + \frac{1}{8} + \cdots + \frac{1}{2^n} \]
for $n \in \mathbb{N}$. Then prove that conjecture using mathematical induction. What is the base case? What is the inductive hypothesis? What are you going to try to prove after you assume the inductive hypothesis? 

		\item Calculate the values of
		\[ \frac{1}{1 \cdot 2} + \frac{1}{2 \cdot 3} \qquad \frac{1}{1 \cdot 2} + \frac{1}{2 \cdot 3} + \frac{1}{3 \cdot 4} \qquad \frac{1}{1 \cdot 2} + \frac{1}{2 \cdot 3} + \frac{1}{3 \cdot 4} + \frac{1}{4 \cdot 5}\]
	\end{enumerate}
	
	\end{description}
	
\end{enumerate}


\begin{description}
		\item[Proposition 0:] (\emph{Example}) For each natural number $n$, 
		\[ 1 + 2 + 3 + \cdots + n = \frac{n(n+1)}{2} \]
		\begin{itemize}
			\item For the base case, I will prove: $1$ equals $\frac{1(1+1)}{2}$ (showing left side of the equation equals the right side when $n=1$)
 			
			\item For the inductive step, I will \textbf{assume}: That for some $k \in \mathbb{N}$, 
			\[ 1 + 2 + 3 + \cdots + k = \frac{k(k+1)}{2} \]
		
			\item For the inductive step, I will \textbf{prove}: 
			\[ 1 + 2 + 3 + \cdots + (k+1) = \frac{(k+1)(k+1+1)}{2} \]
			
					\end{itemize}
	
	
		\item[Proposition 1:] For each natural number $n$, 
		\[ 2 + 5 + 8 + \cdots + (3n-1) = \frac{n(3n+1)}{2} \]
		\begin{itemize}
			\item For the base case, I will prove: 
			\vspace{0.5in}
			
			\item For the inductive step, I will \textbf{assume}: 
			\vspace{0.5in}
			\item For the inductive step, I will \textbf{prove}: 
			\vspace{0.5in}
		\end{itemize}
		
		\item[Proposition 2:] For each natural number $n$, $4^n \equiv 1 \pmod 3$. 
		\begin{itemize}
			\item For the base case, I will prove: 
			\vspace{0.5in}
			
			\item For the inductive step, I will \textbf{assume}: 
			\vspace{0.5in}
			\item For the inductive step, I will \textbf{prove}: 
			\vspace{0.5in}
		\end{itemize}
		
		\item[Proposition 3:] For each natural number $n$, $x-y$ divides $x^n - y^n$. 
		\begin{itemize}
			\item For the base case, I will prove: 
			\vspace{0.5in}
			
			\item For the inductive step, I will \textbf{assume}: 
			\vspace{0.5in}
			\item For the inductive step, I will \textbf{prove}: 
			\vspace{0.5in}
		\end{itemize}
		
		\item[Proposition 4:] For each natural number $n$, $3$ divides $n^3 + 23n$. 
		\begin{itemize}
			\item For the base case, I will prove: 
			\vspace{0.5in}
			
			\item For the inductive step, I will \textbf{assume}: 
			\vspace{0.5in}
			\item For the inductive step, I will \textbf{prove}: 
			\vspace{0.5in}
		\end{itemize}
		
		\item[Proposition 5:] Suppose we draw $n$ straight lines in the plane (which may cross but not lie coincident with each other), dividing it into a number of regions. Then for every natural number $n$, we may color each region either red or blue in such a way that no two neighboring regions have the same color. 	
		\begin{itemize}
			\item For the base case, I will prove: 
			\vspace{0.5in}
			
			\item For the inductive step, I will \textbf{assume}: 
			\vspace{0.5in}
			\item For the inductive step, I will \textbf{prove}: 
			\vspace{0.5in}
		\end{itemize}
		
\end{description}


\subsection*{Part 2: Group work (2 points)}

After time is called, hand in your individual work to the instructor. Then get together with your group and check each others' work. Then, \textbf{make an attempt at a proof of one of the propositions labelled 1--5 above (your choice)}. \emph{You do not have to complete a proof today}, but you do need to make a good-faith effort to begin a proof using the setup that you worked out in Part 1. If you make it through the end of the proof, you'll be given some bonus credit for today. Hand in a neat sketch of your work in progress at the end of the session. 


% 
% \end{enumerate}
% 
% 
% \vfill
% 
% \section*{Extra Problem}
% 
% If your group completes the Problem of the Day successfully with time remaining, try this: 
% 
% \begin{proposition}
% 	The number $\sqrt{5}$ is irrational. 
% \end{proposition}
% 
% What's a lemma you might need for this proof? Make sure you prove that as part of your work. \\
% 
% Can this method of proof be extended to prove that $\sqrt{6}$ is irrational? Why or why not? \\
% 
% 
% \emph{Successful completion of each additional problem will add 1 point to both your Class Work score and the overall Class Work total. However, you must attain a score of at least 8/10 on the main Class Work to receive these points. (That is, you've completed all the problems and they are mostly correct.) }	
% 
% \section*{Futher practice}
% For more practice, please try the following exercises from \S3.5, pages 153---157: 
% \begin{itemize}
% 	\item 6, 7, 11, 12, 17, 18. 
% \end{itemize}
% \textbf{These are all optional. }
% But if you submit writeups of one or more of these that represent good-faith efforts to complete the problem correctly, I will give you feedback on whatever you submit. 
% 
% \newpage
% 
% \begin{center}
% 	\begin{Large}
% 		Writeup for Class Work on \S3.5 (October 2)
% 	\end{Large}
% 	
% \end{center}
% 
% \noindent
% \textbf{Lemma 1:} For all integers $a$, if $3$ divides $a^2$, then $3$ divides $a$. \\
% 
% \noindent
% \emph{Proof}: 
% 
% \vspace{5in}
% 
% 
% \noindent
% \textbf{Proposition 1:} The number $\sqrt{3}$ is irrational. \\
% 
% \noindent
% \emph{Proof}:

\end{document}