\documentclass[11pt]{article}

\pagestyle{empty}                       %no page numbers
\thispagestyle{empty}                   %removes first page number
\setlength{\parindent}{0in}               %no paragraph indents

\usepackage{fullpage}
\usepackage[tmargin = 0.5in, bmargin = 1in, hmargin = 1in]{geometry}     %1-inch margins
\geometry{letterpaper}                  
\usepackage{graphicx}
\usepackage{amssymb}

% Default packages
\usepackage{latexsym}
\usepackage{amsfonts}
\usepackage{amsmath}
\usepackage{amsthm}
\usepackage{palatino}
\usepackage{hyperref}
\usepackage{multicol}
\usepackage{fancyhdr}
\usepackage{enumitem}

\def\pageturn{\vfill
\begin{flushright}
	\begin{small}
		Continued $\rightarrow$
	\end{small}
\end{flushright}
\newpage}

\newtheorem{proposition}{Proposition}
\newtheorem{lemma}{Lemma}


\def\ra{\rightarrow}

\begin{document}
	
	\thispagestyle{empty}
	\renewcommand{\headrulewidth}{0.0pt}
	\thispagestyle{fancy}
	\lhead{Prof. Talbert}
	\chead{MTH 210: Communicating in Mathematics}
	\rhead{October 2, 2012}
	\lfoot{}
	\cfoot{}
	\rfoot{}	
	
	\vspace*{0in}

	
\begin{center}
	\begin{Large}
		Writeup for Class Work on \S3.5 (October 2)
	\end{Large}
	
\end{center}

\noindent
\textbf{Lemma 1:} For all integers $a$, if $3$ divides $a^2$, then $3$ divides $a$. \\

\noindent
\emph{Proof}: We will prove the contrapositive. So assume $3$ does not divide $a$, and we will show that $3$ does not divide $a^2$. Since $3$ does not divide $a$, the Division Algorithm gives us two cases to consider: (1) the remainder obtained when dividing $a$ by $3$ is $1$, and (2) the remainder obtained when dividing $a$ by $3$ is $2$. 

\begin{description}
	\item[Case 1:] Suppose the remainder obtained when dividing $a$ by $3$ is $1$. Then there exists an integer $q$ such that $a = 3q + 1$. We want to show that $3$ does not divide $a^2$, so let us get an expression for $a^2$ first: 
	\begin{align*}
		a^2 &= (3q + 1)^2 \\
		&= 9q^2 + 6q + 1 \\
		&= 3(3q^2 + 2q) + 1
	\end{align*}
From the last line, closure of the set of integers under multiplication and addition tells us that $3q^2 + 2q$ is an integer, which we will call $x$. So we can write $a^2$ as $3x + 1$ with $x$ an integer. This means the remainder obtained when dividing $a^2$ by $3$ is $1$, and hence $3$ does not divide $a^2$. 
	\item[Case 2:] Suppose the remainder obtained when dividing $a$ by $3$ is $2$. Then there exists an integer $q$ such that $a = 3q + 2$. We want to show that $3$ does not divide $a^2$, so get an expression for $a^2$ : 
	\begin{align*}
		a^2 &= (3q + 2)^2 \\
		&= 9q^2 + 12q + 4 \\
		&= 3(3q^2 + 4q + 1) + 1
	\end{align*}
Closure of the set of integers under multiplication and addition tells us that $3q^2 + 4q + 1$ is an integer, which we will call $y$. So we can write $a^2$ as $3y + 1$ with $y$ an integer. This means the remainder obtained when dividing $a^2$ by $3$ is $1$, and hence $3$ does not divide $a^2$.
\end{description}
Since $3$ does not divide $a^2$ in either case, the lemma is proven. \hfill $\blacksquare$

\bigskip

\noindent
\textbf{Proposition 1:} The number $\sqrt{3}$ is irrational. \\

\noindent
\emph{Proof}: For a contradiction, suppose $\sqrt{3}$ is rational. Then there exist integers $a,b$ such that 
\[ \sqrt{3} = \frac{a}{b} \]
and the fraction $a/b$ is in lowest form. Square both sides to get: 
\[ 3 = \frac{a^2}{b^2} \]
Clearing fractions gives us 
\begin{equation}\label{eqn}
	3b^2 = a^2
\end{equation}
From (\ref{eqn}) we see that $3$ divides $a^2$. By Lemma 1, this means $3$ divides $a$, so write $a = 3k$ for some integer $k$. Substituting this expression back into (\ref{eqn}) gives: 
\begin{equation}\label{eqn2}
	3b^2 = (3k)^2 = 9k^2
\end{equation}
From (\ref{eqn2}) we have $b^2 = 3k^2$. This means $3$ divides $b^2$ and so by Lemma 1, $3$ divides $b$. But this is a contradiction, because now $a$ and $b$ share a common factor of $3$, which violates our assumption that $a/b$ is in lowest form. Therefore it cannot be the case that $\sqrt{3}$ is rational; hence $\sqrt{3}$ is irrational as desired. \hfill $\blacksquare$
\end{document}