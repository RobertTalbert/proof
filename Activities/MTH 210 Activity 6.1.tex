\documentclass[11pt]{article}

\pagestyle{empty}                       %no page numbers
\thispagestyle{empty}                   %removes first page number
\setlength{\parindent}{0in}               %no paragraph indents

\usepackage{fullpage}
\usepackage[tmargin = 0.5in, bmargin = 1in, hmargin = 1in]{geometry}     %1-inch margins
\geometry{letterpaper}                  
\usepackage{graphicx}
\usepackage{amssymb}

% Default packages
\usepackage{latexsym}
\usepackage{amsfonts}
\usepackage{amsmath}
\usepackage{amsthm}
\usepackage{palatino}
\usepackage{hyperref}
\usepackage{multicol}
\usepackage{fancyhdr}
\usepackage{enumitem}

\def\pageturn{\vfill
\begin{flushright}
	\begin{small}
		Continued $\rightarrow$
	\end{small}
\end{flushright}
\newpage}

\newtheorem{proposition}{Proposition}
\newtheorem{lemma}{Lemma}


\def\ra{\rightarrow}

\begin{document}
	
	\thispagestyle{empty}
	\renewcommand{\headrulewidth}{0.0pt}
	\thispagestyle{fancy}
	\lhead{Prof. Talbert}
	\chead{MTH 210: Communicating in Mathematics}
	\rhead{October 29, 2012}
	\lfoot{}
	\cfoot{}
	\rfoot{}	
	
	\vspace*{0in}

		\begin{center}
			\begin{large}
			\textbf{Class Work: Introduction to Functions} \\
			\end{large}
			This is a full-time activity worth 10 points. 
			
		\end{center}
		

\section*{Problems of the Day}

\begin{enumerate}
	\item Let $d: \mathbb{N} \rightarrow \mathbb{N}$ be the function defined as follows: For each $n \in \mathbb{N}$, $d(n)$ is the number of natural number divisors of $n$. For example, $d(6) = 4$ because $6$ has $4$ natural number divisors ($1$, $2$, $3$, and $6$). Similarly, $d(81) = 5$ because $81$ has $5$ natural number divisors (what are they?). 
	\begin{enumerate}
		\item Calculate $d(k)$ for each natural number $k$ from 1 through 12.
		\item Find all the preimages of the number 1. (Review the definition of ``preimage'' if needed.) 
		\item Find all the preimages of the number 2. 
		\item Is the following statement true, or false? Justify your conclusion: 
		\begin{center}
			If $m,n \in \mathbb{Z}$ and $m \neq n$, then $d(m) \neq d(n)$. 
		\end{center}
		
		\item Calculate $d(2^k)$ for $k=1, 2, 3, 4, 5, 6$. Based on your results, make a conjecture for a formula for $d(2^k)$ where $k$ is a nonnegative integer. 
		
		\item Is the following statement true, or false? Justify your conclusion: 
		\begin{center}
			For each $n \in \mathbb{N}$, there exists a natural number $m$ such that $d(m) = n$. 
		\end{center}
		If this statement is true, what does it mean about the relationship between the codomain of $d$ and the range of $d$? 
		
	\end{enumerate}
	
	\item Now define a function $S$ that accepts a natural number as input and produces the set of its natural number divisors as output. For example, $S(6) = \{ 1, 2, 3, 6\}$ and $S(81) = \{ 1, 3, 9, 27, 81 \}$. 
	\begin{enumerate}
		\item What is the domain of this function? What is the codomain of this function? (Careful with the second question.) 
		\item Determine $S(k)$ for $k = 1, 2, \dots, 10$ and then for three other values of $k$. 
		\item Is the following statement true, or false? Justify your conclusion: 
		\begin{center}
			If $m,n \in \mathbb{Z}$ and $m \neq n$, then $S(m) \neq S(n)$. 
		\end{center}
		
		\item Is the range of $S$ equal to the codomain? Why or why not? 
	\end{enumerate}
	
	
\end{enumerate}

\section*{Parameters}

If your group finishes your work, please hand it in at the end of class. If all groups finish by the end of class, we will take time to debrief the solutions to one or more of these. 

\end{document}