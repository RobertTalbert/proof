\documentclass[11pt]{article}

\pagestyle{empty}                       %no page numbers
\thispagestyle{empty}                   %removes first page number
\setlength{\parindent}{0in}               %no paragraph indents

\usepackage{fullpage}
\usepackage[tmargin = 0.5in, bmargin = 1in, hmargin = 1in]{geometry}     %1-inch margins
\geometry{letterpaper}                  
\usepackage{graphicx}
\usepackage{amssymb}

% Default packages
\usepackage{latexsym}
\usepackage{amsfonts}
\usepackage{amsmath}
\usepackage{amsthm}
\usepackage{palatino}
\usepackage{hyperref}
\usepackage{multicol}
\usepackage{fancyhdr}
\usepackage{enumitem}

\def\ra{\rightarrow}

\begin{document}
	
	\thispagestyle{empty}
	\renewcommand{\headrulewidth}{0.0pt}
	\thispagestyle{fancy}
	\lhead{Prof. Talbert}
	\chead{MTH 210: Communicating in Mathematics}
	\rhead{September 12, 2012}
	\lfoot{}
	\cfoot{}
	\rfoot{}	
	
	\vspace*{0in}

		\begin{center}
			\begin{large}
			\textbf{Class Work: Quantifiers and Negations} \\
			\end{large}
		\end{center}
		

\begin{enumerate}
	
	\item Each of the following universally quantified statements is false. Give a specific counterexample to show this, and then write the negation of the statement in English (without using any symbols for quantifiers). 
	\begin{enumerate}
		% \item $(\forall x \in \mathbb{R})(x^2 > 0)$
		\item $(\forall m \in \mathbb{Z}) \left( \dfrac{m}{3} \in \mathbb{Z} \right)$
		\item $(\forall a \in \mathbb{Z})(\sqrt{a^2} = a)$
	\end{enumerate}
	
	
	\item For each of the following statements, write the statement as an English sentence that does not use quantifier symbols. Then write the negation of the statement that does use quantifier symbols. Finally, write the negation in English without quantifier symbols. 
	\begin{enumerate}
		\item $(\exists x \in \mathbb{Q})(x > \sqrt{2})$
		\item $(\forall x \in \mathbb{Z})($x$ \, \text{is even or} \, x \, \text{is odd})$
		\item $(\forall x \in \mathbb{Z})(\text{If $x^2$ is odd, then $x$ is odd})$
	\end{enumerate}
	
	\item Let $\mathbb{Z}^*$ be the set of all nonzero integers. Here's a statement with two quantifiers: 
	\begin{center}
		For each $x \in \mathbb{Z}^*$, there exists a $y \in \mathbb{Z}^*$ such that $xy = 1$. 
	\end{center}
	\begin{enumerate}
		\item Is this statement true or false? Why? 
		\item Write the negation of this statement in English with no quantifier symbols. 
	\end{enumerate}
	
	
\end{enumerate}
	
% \subsection*{If you finish all of these...}
% 
% 
% In Section 1.1, we studied some of the closure properties of the standard number systems. We can extend this idea to other sets of numbers: 
% \begin{itemize}
% 	\item We say that a set $A$ of numbers is closed under addition provided that whenever $x,y \in A$, the number $x+y \in A$. 
% 	\item We say that a set $A$ of numbers is closed under multiplication provided that whenever $x,y \in A$, the number $xy \in A$.
% \end{itemize}
% Which of the following sets is closed under addition? Which is closed under multiplication? Explain how you know in each case with either a proof or a counterexample. 
% 	\begin{enumerate}
% 		\item $\{ 1, 4, 7, 10, 13, \cdots \}$ 
% 		\item $\{ 3n \, | \, n \in \mathbb{Z} \}$
% 		\item ${\left\{ \left. \dfrac{1}{2^n} \, \right| \, n \in \mathbb{N}\right\}}$
% 	\end{enumerate}
% 
% 
% Successful completion of this additional problem will add 2 points to both your Class Work score and the overall Class Work total. However, you must attain a score of at least 8/10 on the main Class Work to receive these points. (That is, you've completed all the problems and they are mostly correct.) 	
	
\subsection*{Specifications}

Please hand in a clean copy of your work by the end of your class period. This copy should be written up on paper neatly and in an organized way. All members of the group working on problems should add their names to the paper. Do not include the paper with the problems on it. 

All groups are expected to submit their work by the end of class. I have the right to grant extensions if the majority of groups are working productively but still not completing the problems on time. \emph{However}: Don't expect extensions. Work as if the end of class were a hard deadline. 
\end{document}