\documentclass[11pt]{article}

\pagestyle{empty}                       %no page numbers
\thispagestyle{empty}                   %removes first page number
\setlength{\parindent}{0in}               %no paragraph indents

\usepackage{fullpage}
\usepackage[tmargin = 0.5in, bmargin = 1in, hmargin = 1in]{geometry}     %1-inch margins
\geometry{letterpaper}                  
\usepackage{graphicx}
\usepackage{amssymb}

% Default packages
\usepackage{latexsym}
\usepackage{amsfonts}
\usepackage{amsmath}
\usepackage{amsthm}
\usepackage{palatino}
\usepackage{hyperref}
\usepackage{multicol}
\usepackage{fancyhdr}
\usepackage{enumitem}

\def\ra{\rightarrow}

\begin{document}
	
	\thispagestyle{empty}
	\renewcommand{\headrulewidth}{0.0pt}
	\thispagestyle{fancy}
	\lhead{Prof. Talbert}
	\chead{MTH 210: Communicating in Mathematics}
	\rhead{September 19, 2012}
	\lfoot{}
	\cfoot{}
	\rfoot{}	
	
	\vspace*{0in}

		\begin{center}
			\begin{large}
			\textbf{Class Work: More Methods of Proof} \\
			\end{large}
		\end{center}


\begin{enumerate}

\item Prove that for every positive real number $x$, if $x$ is irrational, then $\sqrt{x}$ is irrational. (\emph{Hint}: What's the contrapositive of this statement? Also, make sure you know the precise definition of a rational number before you start proving anything.) 

\begin{proof}
	We will prove this statement by proving the contrapositive. That is, we will show that for every positive real number $x$, if $\sqrt{x}$ is rational, then $x$ is rational. To that end, suppose $x$ is a positive real number such that $\sqrt{x}$ is rational. Then by definition there exist integers $a,b$ with $b \neq 0$ such that 
	\begin{equation}\label{rationalsqrt}
		\sqrt{x} = \frac{a}{b}
	\end{equation}
	Squaring both sides of (\ref{rationalsqrt}) gives 
	\[ x = \frac{a^2}{b^2} \]
	Since $a,b \in \mathbb{Z}$ with $b \neq 0$, we know by the closure of the set of integers under multiplication that $a^2, b^2 \in \mathbb{Z}$. We also know that $b^2 \neq 0$ since $b \neq 0$. Therefore $x$ is rational, which is what we wanted to show. 
\end{proof}



\item Suppose we have a right triangle whose hypotenuse has length $c$, and the lengths of the other sides are $a$ and $b$. Prove that this right triangle is isosceles if and only if its area equals $\dfrac{1}{4}c^2$. 

\begin{proof}
	First assume that the right triangle in question is isosceles. Then by definition, $b = a$. We want to show that the area of the triangle is $\dfrac{1}{4}c^2$. The area of the triangle is half the base times the height. In our case, the base and height are equal, so: 
	\begin{equation}\label{area1}
		\text{Area} = \frac{1}{2}ab = \frac{1}{2}a^2
	\end{equation}
Since this triangle is a right triangle, we can use the Pythagorean Theorem to relate the hypotenuse and side lengths: 
\begin{equation}\label{pt}
	c^2 = a^2 + b^2 = a^2 + a^2 = 2a^2
\end{equation}
Solving (\ref{pt}) for $a^2$ gives 
\[ a^2 = \frac{1}{2}c^2 \]
Substituting this expression back into (\ref{area1}) gives: 
\[ \text{Area} = \frac{1}{2}a^2 = \frac{1}{2}\left(\frac{1}{2}c^2 \right) = \frac{1}{4}c^2 \]
Therefore the area is $\dfrac{1}{4}c^2$, which is what we wanted to show. \\

For the converse, assume that the area of the right triangle is $\dfrac{1}{4}c^2$. We want to show that the triangle is isosceles, that is, that $a=b$. The area of the triangle is equal to half the base times the height, that is: 
\begin{equation}\label{area2}
	\text{Area} = \frac{1}{2}ab
\end{equation}
Since we are assuming that this area also equals $\dfrac{1}{4}c^2$, we can set these two area expressions equal to each other: 
\begin{equation*}\label{area3}
	\frac{1}{2}ab = \frac{1}{4}c^2 
\end{equation*}
Therefore $c^2 = 2ab$ by multiplying both sides of this by $4$. 

The Pythagorean Theorem says that the hypotenuse and side lengths are related by: 
\[ c^2 = a^2 + b^2 \]
Substituting $c^2 = 2ab$ into this and using algebra, we obtain: 
\begin{align*}
	% a^2 + b^2 &= c^2 \\
	a^2 + b^2 &= 2ab \\
	a^2 - 2ab + b^2 &= 0 \\
	(a - b)^2 &= 0
\end{align*}
Since $(a-b)^2 = 0$, it follows that $a-b= 0$, from which we obtain $a=b$ which is what we wanted. 
\end{proof}
	
\end{enumerate}	

\end{document}