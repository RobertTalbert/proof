\documentclass[11pt]{article}

\pagestyle{empty}                       %no page numbers
\thispagestyle{empty}                   %removes first page number
\setlength{\parindent}{0in}               %no paragraph indents

\usepackage{fullpage}
\usepackage[tmargin = 0.5in, bmargin = 1in, hmargin = 1in]{geometry}     %1-inch margins
\geometry{letterpaper}                  
\usepackage{graphicx}
\usepackage{amssymb}

% Default packages
\usepackage{latexsym}
\usepackage{amsfonts}
\usepackage{amsmath}
\usepackage{amsthm}
\usepackage{palatino}
\usepackage{hyperref}
\usepackage{multicol}
\usepackage{fancyhdr}
\usepackage{enumitem}
\usepackage{mathtools}

\def\pageturn{\vfill
\begin{flushright}
	\begin{small}
		Continued $\rightarrow$
	\end{small}
\end{flushright}
\newpage}

\newtheorem{theorem}{Theorem}
\newtheorem{proposition}{Proposition}
\newtheorem{lemma}[theorem]{Lemma}


\def\ra{\rightarrow}

\begin{document}

\begin{lemma}
	For every natural number $n$, 
	\[ [10^n] = \left\{ \begin{array}{rl}
		[1] & \text{if $n$ is even} \\ 
		\left[-1\right] & \text{if $n$ is odd}
	\end{array} \right. \]
where the equivalence classes come from the equivalence relation of integer congruence mod 11. 
\end{lemma}

\begin{proof}
	This is left as an exercise. Hint: Use induction. 
\end{proof}

\bigskip

\begin{theorem}
	If $n$ is a 5-digit integer such that the difference between its odd- and even-numbered digits is divisible by 11, then $n$ is divisible by 11. 
\end{theorem}

\begin{proof}
	Suppose $n$ is is a 5-digit integer such that the difference between its odd- and even-numbered digits is divisible by 11. Write $n$ in its base-10 form: 
	\begin{equation}\label{base10}
		n = d_4 \times 10^4 + d_3 \times 10^3 + d_2 \times 10^2 + d_1 \times 10^1 + d_0 		
	\end{equation}
We are assuming that $11$ divides $(d_4 + d_2 + d_0) - (d_3 + d_1)$, so in terms of equivalence classes under the relation of integer congruence modulo 11, we have:  
	\begin{equation}\label{eleven}
		[(d_4 + d_2 + d_0) - (d_3 + d_1)] = [0]
	\end{equation}
We want to show $11$ divides $n$, which we can do by proving that $[n] = [0]$ under the relation of congruence modulo 11. To this end, take equivalence classes of both sides of (\ref{base10}) mod 11: 
	\begin{equation*}
		[n] = [d_4 \times 10^4 + d_3 \times 10^3 + d_2 \times 10^2 + d_1 \times 10^1 + d_0]
	\end{equation*}
Now using the properties of the modular arithmetic operations $\oplus$ and $\odot$ mod 11, we have: 
\begin{align*}
	[n] &= [d_4 \times 10^4 + d_3 \times 10^3 + d_2 \times 10^2 + d_1 \times 10^1 + d_0] \\
	&= ([d_4] \odot [10^4]) \oplus ([d_3] \odot [10^3]) \oplus ([d_2] \odot [10^2]) \oplus ([d_1] \odot [10^1]) \oplus [d_0] 
\end{align*}
By the Lemma, we can substitute for the classes of powers of 10: 
\begin{align*}
	[n] &= ([d_4] \odot [10^4]) \oplus ([d_3] \odot [10^3]) \oplus ([d_2] \odot [10^2]) \oplus ([d_1] \odot [10^1]) \oplus [d_0] \\
	&= ([d_4] \odot [1]) \oplus ([d_3] \odot [-1]) \oplus ([d_2] \odot [1]) \oplus ([d_1] \odot [-1]) \oplus [d_0] \\
	&= [d_4] \oplus [-d_3] \oplus [d_2] \oplus [-d_1] \oplus [d_0] \\ 
	&= [d_4 + d_2 + d_0 - (d_3 + d_1)] \\
	&= [0]
\end{align*}
The last line is true because of (\ref{eleven}) above. Therefore $[n] = [0]$, so $11$ divides $n$ as desired. 
\end{proof}

\end{document}