\documentclass[11pt]{article}

\pagestyle{empty}                       %no page numbers
\thispagestyle{empty}                   %removes first page number
\setlength{\parindent}{0in}               %no paragraph indents

\usepackage{fullpage}
\usepackage[tmargin = 0.5in, bmargin = 1in, hmargin = 1in]{geometry}     %1-inch margins
\geometry{letterpaper}                  
\usepackage{graphicx}
\usepackage{amssymb}

% Default packages
\usepackage{latexsym}
\usepackage{amsfonts}
\usepackage{amsmath}
\usepackage{amsthm}
\usepackage{palatino}
\usepackage{hyperref}
\usepackage{multicol}
\usepackage{fancyhdr}
\usepackage{enumitem}


\def\ra{\rightarrow}

\begin{document}
	
	\thispagestyle{empty}
	\renewcommand{\headrulewidth}{0.0pt}
	\thispagestyle{fancy}
	\lhead{Prof. Talbert}
	\chead{MTH 210: Communicating in Mathematics}
	\rhead{September 5, 2012}
	\lfoot{}
	\cfoot{}
	\rfoot{}	
	
	\vspace*{0in}

		\begin{center}
			\begin{large}
			\textbf{Class Activity: Statements and Logical Operators} \\
			\end{large}
		\end{center}
		
This is a FULL-TIME activity worth 10 points. Please work on this INDIVIDUALLY for the first half of the time period and complete as much as you can. Then you'll be put into groups to ask questions and then combine your ideas into a single group writeup for the remaining time. 

\begin{enumerate}
	\item Each of the following is a conditional statement written in something other than the traditional ``if-then'' form. Rewrite each in ``if-then'' form. 
	\begin{enumerate}
		\item The integer $n^2$ is even whenever the integer $n$ is even. 
		\item The integer $n^2$ being odd implies that the integer $n$ is odd. 
		\item Given two points, there is exactly one line between those two points. 
	\end{enumerate}
	\item Construct a truth table for the statement $(\neg Q) \rightarrow (\neg P)$. Does this truth table look similar to another one you've seen? 
	\item Construct truth tables for the statements $\neg(P \vee Q)$ and $(\neg P) \vee Q$. Are the truth tables the same? Do parentheses matter when negating a statement? 
	\item Use a truth table to show that the statement
	\[ [(P \rightarrow Q) \wedge (Q \rightarrow R)] \rightarrow (P \rightarrow R) \]
	is a tautology. (Note that this really is a tautology, so if you don't get a tautology from your truth table, something's wrong with your truth table.) 
\end{enumerate}	
	
\subsection*{If you finish all of these...}

Use a truth table to show that 
\[ \neg [((P \rightarrow Q) \wedge P) \rightarrow Q] \]
is a contradiction. \\

Successful completion of this additional problem will add 2 points to both your Class Work score and the overall Class Work total. However, you must attain a score of at least 8/10 on the main Class Work to receive these points. (That is, you've completed all the problems and they are mostly correct.) 	
	
\subsection*{Specifications}

Please hand in a clean copy of your work by the end of your class period. This copy should be written up on paper neatly and in an organized way. All members of the group working on problems should add their names to the paper. Do not include the paper with the problems on it. 

All groups are expected to submit their work by the end of class. I have the right to grant extensions if the majority of groups are working productively but still not completing the problems on time. \emph{However}: Don't expect extensions. Work as if the end of class were a hard deadline. 
\end{document}