\documentclass[11pt]{article}

% \setlength{\oddsidemargin}{-.35in}
% %\setlength{\evensidemargin}{-.5in}
% \setlength{\textwidth}{7.2in}
% \setlength{\topmargin}{-0.95in}
% \setlength{\textheight}{9.95in}

\usepackage{fullpage}

\usepackage{palatino}
\usepackage{latexsym}
\usepackage{amsfonts}
\usepackage{amsmath}
\usepackage{amsthm}
\usepackage{amssymb}
\usepackage{graphics}
\usepackage{enumitem}
%\usepackage{enumerate}

\usepackage{hyperref} 
\hypersetup{colorlinks=true, linkcolor=blue,  anchorcolor=blue,  
citecolor=blue, filecolor=blue, menucolor=blue, pagecolor=blue,  
urlcolor=blue} 

\newcommand{\ds}{\ensuremath{\displaystyle}}
\newcommand{\nind}{\noindent}

\pagestyle{empty}

\begin{document}
		
\hfill Prof. Talbert
\begin{center}
\large{MTH 210-01/02, Fall 2012 \\
Proof Portfolio Information}
\end{center}

\section{Project Description}

A major goal of MTH 210 is to learn how to solve problems whose solutions involve well-written and convincing mathematical arguments. This goal involves fusing together a mastery of mathematical techniques and skill with writing arguments in a clear and compelling way. The primary tool we will use to build your problem-solving skills, assess your work at problem-solving and writing, and to showcase your progress and mastery is the \textbf{Proof Portfolio}. \\

Over the course of the semester, each student will receive ten problems whose solutions involve crafting coherent, convincing, and mathematically correct arguments, also known as \emph{proofs}. These problems will be assigned at regular intervals; the first three problems will be given on the first day of class, with new problems being added to the list at the rate of about once per week from there on, up until about two weeks from the end of the semester. These problems are sometimes difficult and always require careful thought, lots of attention, and lots of time to think and make mistakes. You will be working on these problems individually outside of class, writing up drafts of your solutions, submitting those drafts, receiving feedback on your drafts, and repeating this draft/feedback/revision cycle on each problem until you have a completed solution. \\

By the end of the semester, you will have a complete package of ten proof-oriented problems and formal, correct, professionally-typeset solutions to them.  \\

\section{Drafts and Draft Submissions}

A \emph{draft} of a solution is a typeset, formal version of what you believe to be a good proof. Scratch work, sketches of ideas, or solutions with contain large gaps that you know are there are not considered to be acceptable drafts. During the semester, you will be submitting multiple drafts of portfolio problems to me; I will provide you with feedback on those drafts, and you'll use the feedback to improve your work until a final product is reached. This draft-revision process is a central element of all Supplemental Writing Skills courses and an essential component of any writing process, mathematical or otherwise. \\

To support timely submission of drafts and a distribution of the workload that doesn't crush you, you'll be given a series of deadlines for submitting drafts throughout the semester. Those deadlines will be explained in the ``Rules, Structure, and Expectations'' section below and placed on the course calendar for reference. You will be allowed up to two drafts on each problem and will receive helpful feedback on each draft. 
\\

You will type up your drafts and submit them to me electronically as PDF files by attaching them to an email and sending to a special email address set up specifically for this course: 
\begin{center}
	\texttt{talbert.mth210.01@gmail.com} \ or \  \texttt{talbert.mth210.02@gmail.com} 
\end{center}
Use the ``01'' email address if you are in section 01 (10:00--10:50) and the ``02'' address if you are in section 02 (11:00--11:50).


Once I receive your drafts, I will review them and offer feedback on them, then send the annotated PDF to you through email. You should then perform any necessary revisions before resubmitting. Each problem is to be submitted twice in draft form before a final grade is awarded. 

\section{Rules, Structure, and Expectations}

\begin{enumerate}[itemsep=0pt]
	\item This is an \emph{independent} project that you should view as a semester-long take home examination.  You may not discuss the problems with anyone except your current MTH 210 instructor, Robert Talbert.  This means you cannot talk to other students about
	your solutions (or even which problems you are choosing to work on), nor can you ask for help from the Math Lab tutors
	or other professors, nor from any person in an online setting.  Violation of this policy is grounds for failure of the course.  Please refer to the syllabus addendum on plagiarism and academic honesty.
	\item You may not refer to any sources other than the textbook for this course.  This includes a prohibition against looking things up on the Internet unless you are directed specifically to do so in the context of a problem.  If you think that you need some background material or a definition from another source then you may ask me for permission, and if granted then you may look up the necessary material and include it with a footnote in your proof. Violation of this policy is grounds for failure of the course.  Please refer to the syllabus addendum on plagiarism and All drafts, both preliminary and final, must be typeset and follow the  guidelines for mathematical writing that are in our textbook and on the handout that is available on the course web page.  Guidance on typesetting issues is provided later in this document. honesty.
	\item You are expected to electronically submit  2 preliminary drafts (draft 1 and draft 2) of each of the 10 portfolio problems.  The drafts are given provisional grades that help you understand the current state of your work.  The third draft of any portfolio problem is final; normally, your third draft will be submitted in hard copy format in your final portfolio during the last week of the term.
	\emph{You are welcome -- indeed, encouraged! -- to discuss your ideas in person with me on any given problem at any stage:  before or after submitting either draft 1 or draft 2}.
	\item Note that the provisional grades assigned to drafts are just that: Provisional. They will be recorded in a spreadsheet for my reference, but they will not count for or against your grade. If you submit two drafts but fail to submit the final draft by the end of the semester, you will receive no credit for the problem. And note that it is possible to receive a lower grade on the final draft than the provisional grades you received on earlier drafts on the same problem, if you make changes to your solution that lower its overall quality. 
	
	\item While you are required to submit two drafts per week, you may submit up to one more, should you wish to work ahead.  \textbf{No more than three drafts may be submitted in any single week}. Weeks begin at 6am on Monday mornings and end at 5:59 am on Mondays. Submissions will be made electronically, and thus can be turned in at any time, but there are two deadlines each week: one is at 6am on Wednesday morning, the other is 6am Friday morning. There are exceptions to this rule: In the week before a test, there is only one deadline --- 6am Wednesday. You have the Fridays of those two weeks ``off'' to review and prepare for your tests. Also, there are  no drafts due during the week of Thanksgiving Recess (November 19--23). 

	Here is a summary of how the deadlines will function:  Starting on \textbf{Friday, September 7}, a first draft of a proof is due by 6am.  From there forward, on each subsequent Wednesday and Friday (with the exceptions mentioned above), by 6 am you will need to submit at least one other draft. This may be the first draft of a new problem or a second draft of a previously-submitted problem. (Typically, the best sequence to follow is to alternate between first and second drafts.) The problems need not be completed in the order that I number them, and it doesn't matter whether you turn in a first or second draft by Wednesday or Friday, just as long as one of each is turned in each week.  The last date you will be able to submit a draft is \textbf{Wednesday, November 28 at 6am}. 
	
	\item Each of the ten problems is individually worth 10 points.  In addition, there are 20 required drafts (2 for each of the 10 problems), and you will receive 20/20 for meeting all 20 of the draft deadlines.  For each deadline you miss, there is a 2-point deduction.  \emph{Note well: if you fall behind, you can only ever submit 3 drafts in a given week, so you begin forfeiting opportunities for review by missing deadlines.}

	\item I will always strive to have drafts reviewed and returned to you within 2 business days of their submission; ``business days'' are MTWRF.  So, if you submit a problem by 6am on Wednesday, I will strive to have it back to you by 6am on Friday. If you submit a problem by 6am on a Friday, I will strive to have it back to you by 6am on the following Tuesday. 

	\item I will not discuss portfolio problems in office hours or via email after Friday, November 30.  The final version of proof portfolio (in printed, hard-copy form) is due at the start of your section's final class meeting on \textbf{Friday, December 7}. Only the scores on the final versions will count toward your grade.
	
\end{enumerate}

\section*{Grading}

\noindent The portfolio will be graded on a scale of 120 points.  This 120 point grade counts 30\% of your grade in the course.  
 \\

\noindent Each of the 10 problems will be worth 10 points (for a total of 100 points), and in addition, there will be 20 points possible for meeting the aforementioned draft deadlines. \textbf{Note well:} \emph{This means that the deadline requirements by themselves constitute 5\% of your grade. }For the problems themselves, grades will be awarded according to the following scheme:

\bigskip

\begin{tabular}{r p{5in}}
10 points  &  Mathematically perfect proof; follows all guidelines for writing\\

9 points &  Mathematically correct proof, but minor error (note singular) in writing  (including use of mathematical notation)\\

7 points &  Significant mathematical progress has been made towards a proof\\
& but either the argument has one major error or it is does not yet meet the writing guidelines \\

5 points &  Some significant mathematical progress has been made towards a proof \\ & but there are key errors present in the mathematics or major issues with the written presentation \\

3 points & Evidence of having at least one good idea 
and making an effort to write a formal proof\\

0 points &  Essentially no progress has been made towards a
correct proof; or there are serious fundamental writing errors\\

\end{tabular}

Note that some work may merit points that are between the above values. For example, work that is judged to be on the borderline between a 5 and a 7 may be given a 6. Non-integer point values will not be given. Both grades on provisional drafts and final versions of each problem will be given according to this scale.  Grades on provisional drafts are given to help you see where your progress on the problem stands.  Only grades on final drafts count towards the 120 point total. \\



\section{How to Submit Your Work}

Each solution or proof must be done on a word processor capable of producing the appropriate mathematical symbols and equations.   The document must then be converted to \texttt{.pdf} format for submission. Handwritten work or typeset work that does not employ appropriate mathematical notation (such as writing ``\verb=x^2='' instead of $x^2$) will be given a grade of 0. \\

The mathematical typesetting system \LaTeX\ (pronounced ``LAY-tek'') will be the preferred system for producing typeset mathematics. \LaTeX\ is an industry-standard software for producing technical documents and is both free and well-supported. You will be learning \LaTeX\ as part of the course by way of tutorials, documents, and videos. Resources for learning \LaTeX will be given to you in a separate document. \\

For your first two draft submissions, you may use any software you want to typeset your drafts. Options include \LaTeX, Microsoft Word and its Equation Editor, or Google Documents and its equation editor. Do note that if you use either Word or GDocs, you will need to use the \textbf{Save As...} menu option to convert your work to PDF form. If you are unfamiliar with the Equation Editor in Word, you can begin to learn more at 
\begin{center}
\href{http://www.youtube.com/watch?v=rUrVUhcWLnQ}{\texttt{http://www.youtube.com/watch?v=rUrVUhcWLnQ}}. 
\end{center}

However, neither Word nor GDocs is a good long-term solution for writing technical documents for a number of reasons; \LaTeX\ is. Therefore: 
\begin{enumerate}[itemsep=0pt]
	\item Students who type up either of their first two draft submissions using \LaTeX\ and submit the PDF's will receive $+2$ bonus points per submission on the proof portfolio (for a possible total of $+4$ bonus points). And,
	\item \emph{All} draft submissions starting with the third draft deadline \emph{must} be done in \LaTeX. 
\end{enumerate}
This gives each student about three weeks to learn the rudiments of \LaTeX\ and rewards those who take initiative to learn the system early. \\ 

Each solution or proof for a portfolio problem must be submitted in PDF format to the instructor electronically via email at 
\begin{center}
	\texttt{talbert.mth210.01@gmail.com} \ or \ \texttt{talbert.mth210.02@gmail.com} 
\end{center}
Use the ``01'' email address if you are in section 01 (10:00--10:50) and the ``02'' address if you are in section 02 (11:00--11:50). 

Specifically, email your proofs as attachments to messages, and title the file in the following way:

\begin{center}
  \texttt{LastName Problem n Draft m.pdf}
\end{center}
For example, if Carl Gauss were submitting his second draft of problem 3B, the file would be titled 
\texttt{Gauss Problem 3B Draft 2.pdf}.  In addition, he would use this title as his email subject line:
\begin{center}
  \texttt{subject:} Gauss Problem 3B Draft 2
\end{center}

\noindent Please submit only .pdf files for me to view and comment on.   I will mark up the file with my comments and send it back to you.

\section{Frequently Asked Questions}

Following are some (asked and anticipated) questions about this Portfolio.  The answers to these questions contain some important additional guidelines and expectations for the Portfolio Project.  You should read all of this carefully and follow the instructions accordingly.\\

\nind {\bf What other requirements are there for my Portfolio Problems?}\\

\nind The solution for each problem must be written using complete sentences and according to the writing guidelines specified in the text. It must be typeset, well organized, and easy to read.  Proper grammar, proper sentence and paragraph structure, and correct spelling are necessities.  Papers that do not adhere to these basic requirements will receive a draft score of zero.\\

\nind {\bf What happens if I submit an incorrect or incomplete solution?}\\

\nind Each time you submit a draft, I will return your problem and indicate if it is finished and ready for inclusion in the final portfolio or if it needs more work.  When you submit a solution for review, you are asking the instructor, ``Is this good enough for my Portfolio?''   I will respond with comments and suggestions, as well as a provisional grade, in order to help you to assess your progress.  While provisional grades are given as a guide (and are part of receiving 20 points of credits for drafts), there is no penalty for a grade lower than 10 \emph{until the third and final draft is submitted}.\\

\nind {\bf Can I work with someone else or use sources other than the textbook?}\\

\nind No.  No collaboration is permitted.  The only person you can discuss these problems with is your current MTH 210 instructor, Robert Talbert, and the only resource you may use is the textbook.  Use of other human or intellectual resources is considered plagiarism and is not acceptable. This includes asking questions on Piazza. One of the primary goals of MTH 210 is that you acquire deep personal understanding of proof techniques and the ability to read and write proofs.  Being able to do so independently is essential, and thus this project is an independent endeavor.  \\

\nind {\bf Can I come to your office for help?}\\

\nind Yes!  You are welcome at any time during office hours or when my door is open to discuss questions on the portfolio (or any other aspect of the course).  If my stated times do not suit your schedule, please request an appointment (ideally, at least 24 hours in advance).  There is only one requirement for you when you come to seek help:  \emph{do not come empty-handed}.  By this I mean that you should not come unprepared saying that you ``have no idea where to start.''  Part of learning to write proofs is thinking of possible ways to start, even if those ways turn out
to be wrong.  When you have chosen a problem to work on, start
your scratch work with a list of things that you know which seem
like they might be related.  Write down what you know and what you
need to show, and see if any of your ideas help with even a small
part of this task.  See more on this in ``Making the Most of Office Hours'' at the end of the course syllabus. If you come to office hours without having seriously undertaken the basic groundwork on solving a problem, I will defer meeting with you until a later time when you've had a chance to do so.
\\

\nind {\bf What criteria will be used to judge my proofs?} \\

\nind A proof must be logically and mathematically correct. In addition, it must be written according to the stated guidelines distributed in class (that will also be available from the course web page, or in more complete form in Appendix A of the text on p.~414).  You will get a better feel for how the grading process works by simply submitting drafts for review. \\



% \section{Example of a Well-Written Solution} 
% 
% \begin{center} \underline{\hspace{5in}} \end{center}
% 
% \nind {\bf Conjecture X.}  If  $x$  and  $y$  are real numbers, then   
% $ \displaystyle \frac{x+y}{2} \geq \sqrt{xy}.$
% 
% \begin{center} \underline{\hspace{5in}} \end{center}
% 
% This proposition is false as is shown by the following counterexample\footnote{
% If the proposition is true, your job is to write a complete proof for the proposition.  If it is false, you should provide a counterexample \emph{plus} make reasonable modifications to the stated conjecture so that a new proposition is true.  Then, write a complete proof of this new proposition.}:  Letting $x = -2$ and $y = -2$, observe that it follows that
% 	$$\frac{x+y}{2} = -2 \ \mbox{and} \sqrt{xy} = 2.$$
% In this case, $\frac{x+y}{2} < \sqrt{xy}.$  This shows that the given proposition is false, because our example demonstrates that the hypothesis of the conditional statement can be true, but the conclusion of the conditional statement false.  \hfill $\Box$ \\
% 
% However, based on a large collection of examples where $x$ and $y$ are both positive, it appears that if   $x$  and  $y$  are nonnegative real numbers, then  $\frac{x+y}{2} \geq \sqrt{xy}.$   We will state this as a theorem and prove it.
%  
% \bigskip
% 
% \nind {\bf Theorem:}  If  $x$  and  $y$  are nonnegative  real numbers, then   
% $$\frac{x+y}{2} \geq \sqrt{xy}.$$
% 
% \nind \emph{Proof:}  We assume that  $x$  and  $y$  are positive real numbers.  The goal is to show that $\frac{x+y}{2} \geq \sqrt{xy}.$  Observe that since $x$ and $y$ are real numbers, and $\mathbb{R}$ is closed under subtraction, $(x-y)$ is also a real number.  Further, since the square of any real number is greater than or equal to zero, we know that $(x-y)^2 \geq 0$.  Expanding the left side of this inequality gives us
% $$x^2 - 2xy + y^2 \geq 0.$$
% 
% We now add $4x$ to both sides of this inequality.  This is done so that the left side will become the square of $(x+y)$.  We see that
% $$x^2 - 2xy + y^2 + 4xy \geq 4xy,$$
% and therefore
% $$x^2 + 2xy + y^2 \geq 4xy.$$
% Factoring the lefthand side and dividing both sides by 4, we have
% \begin{equation} \label{Eq1}
% \frac{(x+y)^2}{4} \geq xy.
% \end{equation}
% Since the function $g(x) = \sqrt{x}$ is an increasing function, it will preserve an inequality, hence we can take the square root of both sides of Equation~(1).  Note also that for any real number, $\sqrt{a^2} = |a|$.  Since $x$ and $y$ are both nonnegative, it follows here that $\sqrt{(x+y)^2} = |x+y| = x + y$.  Thus, taking square roots yields
% $$\frac{x+y}{2} \geq \sqrt{xy},$$
% and the theorem has been proved.  In particular, we have shown that if  $x$  and  $y$  are positive  real numbers, then   
% $$\frac{x+y}{2} \geq \sqrt{xy}.$$
% \hfill $\Box$
% 
% 

%%%%%%%%%%%%%%%%%%%%%%%%%%%%%%%%%%%%%%%%%%%%%%%%%%


\end{document}


  