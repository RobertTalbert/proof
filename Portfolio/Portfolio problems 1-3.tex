\documentclass[11pt]{article}

% \setlength{\oddsidemargin}{-.35in}
% %\setlength{\evensidemargin}{-.5in}
% \setlength{\textwidth}{7.2in}
% \setlength{\topmargin}{-0.95in}
% \setlength{\textheight}{9.95in}

\pagestyle{empty}                       %no page numbers
\thispagestyle{empty}                   %removes first page number
\setlength{\parindent}{0in}


\usepackage{fullpage}

\usepackage{palatino}
\usepackage{latexsym}
\usepackage{amsfonts}
\usepackage{amsmath}
\usepackage{amsthm}
\usepackage{amssymb}
\usepackage{graphics}

\usepackage{hyperref} 
\hypersetup{colorlinks=true, linkcolor=blue,  anchorcolor=blue,  
citecolor=blue, filecolor=blue, menucolor=blue, pagecolor=blue,  
urlcolor=blue,pdftitle={Boelkins PPS12}} 

\newcommand{\ds}{\ensuremath{\displaystyle}}
\pagestyle{empty}


\begin{document}
	
\begin{center}
	MTH 210: Communicating in Mathematics \\ 
	Proof Portfolio Problems 1---3 
\end{center}

\framebox{
\parbox{\textwidth}{
\begin{center}
	\textbf{Problem 1} \\ 
	\emph{Choose either Problem 1A or Problem 1B to do. Note that Problem 1B has two parts, and you must do both parts if you choose that one. }
\end{center}



\noindent
\textbf{Problem 1A:} The notions of \textbf{type 0}, \textbf{type 1}, and \textbf{type 2} integers are defined in Exercise 9 of Section 1.2. Suppose that $a$ and $b$ are type 2 integers and look at the integer $a^2 + b^2$. What can you conclude about the type of this integer? That is, is $a^2 + b^2$ always type 0? Or is it always type 1? Or is it always type 2? Or can it sometimes be different types, depending on the specific values of $a$ and $b$? By experimenting with different specific type 2 integers $a$ and $b$, decide which of the four possibilities above is correct. If you believe one of the first three possibilities is correct, write this in the form of a conjecture that looks like: 
\begin{center}
	If $a$ and $b$ are type 2 integers, then $a^2 + b^2$ is a  \underline{\hspace{1in}} integer.
\end{center}
where the blank is filled in with ``type 0'', ``type 1'', or ``type 2'', and then give a formal mathematical proof of this conjecture. On the other hand, if you believe that $a^2 + b^2$ can be different types depending on $a$ and $b$, then say so clearly and then give specific counterexamples that show that $a^2 + b^2$ can be different types. (Remember: $a$ and $b$ must be type 2 integers.) 



\medskip

\textbf{Problem 1B:} This option has two parts. You must do BOTH parts. 
\begin{itemize}
	\item Suppose $a,b,$ and $c$ are integers. What can we say about whether $ab + ac$ is always even or always odd? If you believe that $ab + ac$ must always be even, then write a conjecture to this effect and prove it. Similarly, if you believe $ab + ac$ is odd, write a conjecture to this effect and prove it. If you believe $ab + ac$ can be either even or odd depending on the values of $a,b$ and $c$, then say so and then give specific examples that shows this.
	\item Suppose $a$ is an integer and $b$ and $c$ are both \emph{odd} integers.  What can we say about whether $ab + ac$ is always even or always odd? If you believe that $ab + ac$ must always be even, then write a conjecture to this effect and prove it. Similarly, if you believe $ab + ac$ is odd, write a conjecture to this effect and prove it. If you believe $ab + ac$ can be either even or odd depending on the values of $a,b$ and $c$, then say so and then give specific examples that shows this.
\end{itemize}
}}
	
	
\vfill
\begin{flushright}
	Continued on the back $\rightarrow$
\end{flushright}

\framebox{
\parbox{\textwidth}{
\begin{center}
	\textbf{Problem 2} \\ 
	\emph{Choose either Problem 2A or Problem 2B to do. }
\end{center}


\noindent
\textbf{Problem 2A}: In this problem, you will in some way use the Pythagorean Theorem, which we will accept to be true without proof:  ``For any right triangle, if its legs have length $a$ and $b$ and its hypotenuse length $c$, then $a^2 + b^2 = c^2$.'' Here's another important term for this problem: a \emph{Pythagorean triple} $(p,q,r)$ is a triple of natural numbers $p < q < r$ such that $p^2 + q^2 = r^2$.  For instance, $(3,4,5)$ is a Pythagorean triple.  Prove or disprove the following conjecture: If $m$ is a natural number and $m \geq 2$, then $(2m, m^2-1, m^2+1)$ is a Pythagorean triple. 

\medskip

\noindent
\textbf{Problem 2B}: Suppose that $f(x) = x^3 + ax^2 + bx + c$ is a cubic polynomial function with $a,b,c$ real numbers and $a^2 > 3b$. Prove or disprove the following conjecture: The $x$-coordinate of the inflection point of $f$ lies halfway between the $x$-coordinates of the two critical points of $f$. 
}}
	
\vspace{1in}


\framebox{
\parbox{\textwidth}{
\begin{center}
	\textbf{Problem 3} \\ 
\emph{Choose either Conjecture 3A or Conjecture 3B to do. If you believe the conjecture is true, give a formal mathematical proof. If you believe the conjecture is false, give a specific counterexample and prove your counterexample works.
	Note that both of these statements are ``if and only if''. If one direction of the biconditional statement is false, give a counterexample for it, but also check to see if the opposite direction is true. If so, give a proof.}
	 
\end{center}

\noindent
\textbf{Conjecture 3A}: For each integer $a$, $a \equiv 3 \ (\textrm{mod}\ 7)$ if and only if $(a^2 + 5a) \equiv 3 \ (\textrm{mod} \ 7)$. 
\medskip

\noindent
\textbf{Conjecture 3B}: For any integer $k$, $k^2 + 4k + 5$ is even if and only if $4 | (k^2 + 2k -1)$.
}}
	
	
\end{document}