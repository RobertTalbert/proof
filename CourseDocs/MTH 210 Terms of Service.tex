\documentclass[11pt]{article}

\pagestyle{empty}                       %no page numbers
\thispagestyle{empty}                   %removes first page number
\setlength{\parindent}{0in}

\usepackage{fullpage}
\usepackage[tmargin = 0.5in, bmargin = 1in, hmargin = 1in]{geometry}     %1-inch margins
\geometry{letterpaper}                  
\usepackage{graphicx}
\usepackage{amssymb}

\usepackage{latexsym}
\usepackage{amsfonts}
\usepackage{amsmath}
\usepackage{amsthm}
\usepackage{palatino}
\usepackage{hyperref}
\usepackage{multicol}
\usepackage{enumitem}

\begin{document}

\begin{center}
	\begin{large}
		MTH 210: Communicating in Mathematics \\
	\end{large}
	Terms of Service 
\end{center}

This document is an addendum to the course syllabus and lists details for all course policies. Please read this document thoroughly and refer back to it. You will be asked to sign a form stating that you have read and understood the policies in this document. 

\section{Absences, class cancellations, and makeup work}

\subsection{Absences and makeup work}

\begin{itemize}[itemsep=0pt]
	\item Students are expected to attend every class meeting unless attendance is impossible. 
	\item Absence from a class without a sufficient reason, communicated to me in a timely manner, will result in the loss of all points for class activities for that day. ``Timely'' means at least one hour before class if possible and no more than 6 hours after class. 
	\item If you miss a class, you will lose all points for activities on that day unless you contact me as soon as possible about the absence, provide documentation verifying the validity of the absence that includes the contact information for a person who can verify the information, and agree to follow a makeup timetable of my choosing. Please note that I will follow up on all such absences. 
	\item The previous point applies to illness particularly. If you are sick enough to miss class, please see Campus Health Services or another health professional and have them provide you with appropriate documentation. Otherwise, you are expected to attend class. 
	\item Requests for late submissions or makeups will only be considered if: (1) the situation is out of the student's control, (2) the student abides by the rules stated above, and (3) the student's reasoning is verified after contacting the person listed in the request.
	\item Late arrivals to class without a sufficient reason will result in a grade of 0 on the clicker quiz for that day if the quiz is completed before arrival. No extensions on time will be given if a student arrives during the quiz. 
\end{itemize}

\subsection{Class cancellations}

\begin{itemize}[itemsep=0pt]
	\item If inclement weather makes it impossible for you to make it to class safely, please contact me well in advance of the absence. Please note the word ``impossible'' in this description. If you \emph{can} come to class through some means, you are expected to do so.
	\item In case of inclement or dangerous weather, please seek information on whether GVSU is open through the regular media and GVSU's web site. Note that GVSU does not necessarily close when Allendale or Grand Rapids public schools close. Please also check your email and Piazza in these circumstances in case weather makes it impossible for the professor to come to campus. 
	\item In the event the professor cannot come to class (for weather, illness, family situation, etc.), the class may not actually be cancelled. Instead, there may be an individual assignment to complete electronically during the class time. Again, always check email for instructions and do not assume class is cancelled.
\end{itemize}


\section{Academic Honesty}

In the interests of honesty, I will disclose that I am using here large chunks of the Academic Dishonesty policy developed by Prof. Matt Boelkins for his MTH 210 sections, with his permission.

\subsection{Definition of academically honest work}

\emph{Academically honest} work by a student is work that authentically reflects the student's understandings, however incomplete, of the work being done. All the major aspects of the work handed in have been done by the student, without collaboration with others (unless otherwise authorized). Use of collaboration that has been authorized has been fully disclosed in the work. 

\subsection{Allowable and non-allowable collaboration}

\begin{itemize}[itemsep=0pt]
	\item You are expected and encouraged to collaborate during the portions of class preparation where you are reading or watching video. 
	\item On the Guided Practice assignments, every step of your work must be one that you understand yourself and that you have generated in thinking about the activities. While you are permitted to collaborate on big ideas and hints with classmates, you should be working alone when you write up your solutions. All such collaboration must occur when your peer is at the same stage of solution as yourself. In addition, in your submission of Guided Practice, you must state the name of the person(s) you collaborated with. You may not have more than two collaborators on any one assignment. Different groups are not allowed to interact with each other. 
	\item Some in-class activities will be designated as Individual activities, and some will be Group activities. In Group activities, you may collaborate freely with the members of your group, but different groups may not interact. On Individual activities, please do all work yourself with no interaction except with the professor. 
	\item You may initiate or engage in a discussion on Piazza regarding any class work in which collaboration is allowed, for example Guided Practice assignments, videos, and Group in-class activities. However, responses to such discussion threads must conform with the second point above: Those responses must not contain ``spoilers'' that provide detailed hints or solutions to the question being asked. All responses must still require the original questioner to think through the problem himself or herself. Posts that exceed those bounds will be removed; repeated posting of such responses will be grounds for academic dishonesty. 
	\item On the Proof Portfolio Project, no collaboration is permitted. The only person you are allowed to discuss these problems with is your instructor, Robert Talbert. As such, every word you write must be your own. Any form of collaboration with a person other than the instructor is considered plagiarism: you are submitting someone else’s work as if it is your own. By carefully reading every word of your work, I will get to know you and your work extremely well this semester. You should be advised that it is often blatantly obvious when a student submits work that is not his own.
	\item No collaboration at all is allowed on quizzes, midterm exams, or the final exam.  
	\item On Guided Practice assignments, Class Work, and the Proof Portfolio Project, the only external resources you may use are our course textbook and your class notes. You are not permitted to go looking for completed solutions to problems in any other texts or resources unless you ask for the professor's permission and receive it. In particular, using the internet is completely off limits for the Proof Portfolio. Evidence of using internet sources in your work will result in a minimum penalty of failure of the assignment.
	\item On any assignment, it is an act of plagiarism to base your work on the efforts of a friend or acquaintance who completed the course in a prior term, even if you give that person credit in your work. Use of such materials in your work this semester is grounds for failure of the course. If you have past exams, past homework assignments, or portfolio proofs from a friend who previously completed MTH 210, please return them immediately. If you yourself are repeating MTH 210, you should talk with me individually about the role these materials can and should play in the course.
\end{itemize}

\subsection{Academic dishonesty in clicker use}

\begin{itemize}[itemsep=0pt]
	\item Students are expected to use only their own clickers in class. 
	\item Attendance for class is recorded separately from the clicker answer data. Therefore, if a student is absent but shows a response for a clicker question, another student is responding for the absent student. This is a violation of academic honesty. 
	\item Therefore any student who does not attend class but who shows clicker responses for that day is automatically guilty of academic dishonesty.
\end{itemize}

\subsection{Consequences for academic dishonesty}

Punishments for academic honesty violations are at the discretion of the instructor and may range from loss of credit all the way to automatic failure of the course and expulsion from the university. For full details on GVSU's academic honesty policy, see the GVSU Student Code (\url{http://gvsu.edu/s/b9)}, especially Section 223.00.


\section{Submission of coursework}

\subsection{Expectations and common rules for student work}

\begin{itemize}[itemsep=0pt]
	\item All work in MTH 210 assumes that you will provide not only a correct answer to the problem or question but also a complete, clear, and correct solution that explains how you got your answer. Unless otherwise specified, an answer without work to back it up will receive no credit; and solutions which do not fully explain the thought process behind the work will not receive full credit. 
	\item Those solutions often are a combination of mathematics and English. Students are expected to abide by basic rules of English usage, such as correct spelling, punctuation, and grammar, when providing written parts of a solution. 
	\item Solutions without English to explain the mathematics are often not as clear as they could be and may lose credit as a result. 
	\item All work in MTH 210 is to be done as cleanly and neatly as possible. Work that is disorganized or messy may lose credit due to lack of clarity and legibility. 
	\item A separate, and much more stringent, set of rules for submissions for Portfolio problems is given in the handout ``Proof Portfolio Information''. 
\end{itemize}

\subsection{Late work}

All work to be handed in will be given a precise time when it is expected. Work is considered ``late'' if it is submitted past that time. Late submissions will generally not be accepted. The exception is if timely submission is impossible, in which case the same rules for makeup work for missed classes (above) apply. 

\section{Technology}

\subsection{The Piazza discussion board}

\begin{itemize}[itemsep=0pt]
	\item All students are expected to join the Piazza discussion board by the second day of class. Students registered as of August 8 received an email with a signup link. All others may sign themselves up at \url{piazza.com/gvsu/fall2012/mth210}.
	\item Piazza will be used as the central location for course announcements, outside-of-class discussions, polls, and other forms of communication. 
	\item Students will be expected to read the Piazza discussion boards at least twice a day, once in the morning and again in the evening, for significant items. The most significant ones will be ``pinned'' so that they are at the top of the board. 
	\item Students are encouraged to engage in discussions on Piazza to help other students (within Academic Honesty bounds; see above). Responses that are particularly helpful (either because they are voted up using the ``good note'' feature or by the professor's judgment) may be given bonus credit. 
	\item Abusive or unhelpful comments will be deleted. Repeated instances of abusive, offensive, or otherwise unhelpful comments will result in the user being removed from the discussion board and possible further action. 
	\item Students may post anonymously, but repeated instances of abusive, offensive, or otherwise unhelpful comments being posted anonymously will result in a revocation of the ability to post anonymously. 
\end{itemize}


\subsection{Blackboard and email}

\begin{itemize}[itemsep=0pt]
	\item All students are expected to check Blackboard (including the class Google calendar) and their email accounts at least twice daily for class updates, once in the morning and once in the evening. 
	\item All grades will be posted to Blackboard. It is your responsibility to maintain a separate set of records and double-check Blackboard for clerical errors. \textbf{All Blackboard grades will be considered final one week after posting them}. If you have a grade discrepancy or a question about a grade, you must bring it to my attention within one week, or else the grade on Blackboard stands. 
%	\item Online homework scores are recorded on WeBWorK and are not posted to Blackboard. Attendance records will be kept separately offline. 
%	\item Students are expected to maintain records of their own grades, separately from Blackboard, in order to double-check all grade postings and calculations. 
\end{itemize}


\subsection{Clicker usage}

We will use clickers almost every day for instructional purposes. Each student is responsible for purchasing a clicker by the second day of class and then bringing it to every class meeting. Clicker records will be used to cross-check attendance records, and if a student has not attempted every clicker question, he or she will be marked absent and lose participation credit. 

\subsection{Personal technology}

\begin{itemize}[itemsep=0pt]
	\item There is no policy banning the use of cell phones, laptops, and other technology in class. Instead, we will insist that all technology use in class must be for assisting in learning the course material or else put away. If this rule is not followed, you may be asked to leave the class, forfeiting your credit for the day.
	\item Please turn all cell phones to silent or vibrate mode, and do not accept or make any calls or text messages during the class meeting unless it is an emergency.
\end{itemize}


\section{Disclaimer}

Additions and changes to this list of policies may take place during the semester. All such changes will announced to the full class during a class meeting or on Blackboard and/or Piazza.

\end{document}