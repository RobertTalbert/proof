\documentclass[11pt]{article}

% Adjust top and side margins
%\setlength{\oddsidemargin}{-.35in}
%%\setlength{\evensidemargin}{-.5in}
%\setlength{\textwidth}{7.0in}
%\setlength{\topmargin}{-0.95in}
%\setlength{\textheight}{9.6in}

\pagestyle{empty}                       %no page numbers
\thispagestyle{empty}                   %removes first page number
\setlength{\parindent}{0in}               %no paragraph indents

\usepackage{fullpage}
\usepackage[tmargin = 0.5in, bmargin = 1in, hmargin = 1in]{geometry}     %1-inch margins
\geometry{letterpaper}                  
\usepackage{graphicx}
\usepackage{amssymb}

% Default packages
\usepackage{latexsym}
\usepackage{amsfonts}
\usepackage{amsmath}
\usepackage{amsthm}
\usepackage{palatino}
\usepackage{hyperref}
\usepackage{multicol}
\usepackage{enumitem}


\begin{document}
	
% Course number, section, name, meetings, semester
\begin{center}
	{\bfseries Syllabus for MTH 210, Sections 01 and 02} \\
	{\bfseries Communicating in Mathematics} \\
	MWF 10:00--10:50 (Sec 01) and 11:00--11:50 (Sec 02), MAK C-1-110  \\
	Fall 2012
\end{center}

% Instructor information
\begin{tabbing}
\textbf{Instructor}: \hspace{0.5in} \= Dr.~Robert Talbert
\hspace{1.4in} \= {\bfseries Email:} \ \ \ \= \href{mailto:talbertr@gvsu.edu}{talbertr@gvsu.edu} \\
\textbf{Office}: \> MAK A-2-168 \> \textbf{Phone}: \> 331-8968 \\
\textbf{Website}: \> \href{http://faculty.gvsu.edu/talbertr}{faculty.gvsu.edu/talbertr} \>  \textbf{Skype}: \> rtalbert235 \\

\\

%Office hours
\textbf{Office hours}: \> MW 2:30--4:00 \end{tabbing}
I am also available by appointment in person, over the phone, or on Skype. I will be on Skype during all posted office hours unless there are technical difficulties. Please also feel free to add me on Google+ (\href{http://gplus.to/rtalbert}{http://gplus.to/rtalbert}) or Twitter (\href{http://www.twitter.com/RobertTalbert}{$@$RobertTalbert}) for other contact opportunities.


% Prerequisites and textbook/materials
\begin{tabbing}
\textbf{Prerequisites}: \hspace{0.3in} \= \parbox[t]{5.25in}{MTH 201 (Calculus I) and WRT 150. A grade of C (not C-) or better is required in WRT 150. Students with a grade of C+ or lower in MTH 201 are at great risk in 210, and should meet with the instructor during the first week.} \\
%\\ 
\\
\textbf{Text}: \> \parbox[t]{5.25in}{\emph{Mathematical Reasoning: Writing and Proof}, Third Edition by Ted Sundstrom. The text is only available as a coursepack from the bookstore.}

\\ \\
\textbf{Blackboard}: \> \parbox[t]{5.25in}{Course documents, assignments, and grades will be housed through the course Blackboard page at \url{http://bb.gvsu.edu}. You will be expected to access the course home page frequently and respond accordingly to announcements and assignments posted there. In addition, it will be essential to be a regular user of email in this course.} \\ \\

\textbf{Discussion forum:} \> \parbox[t]{5.25in}{This term we will be using \textbf{Piazza} for class discussion. The system is highly catered to getting you help fast and efficiently from classmates, the tutors, and myself. Rather than emailing questions to me, I encourage you to post your questions on Piazza. If you have any problems or feedback for the developers, email team@piazza.com.
Find our class page at: \url{https://piazza.com/gvsu/fall2012/mth210}.}. \\ \\


\textbf{Other materials}: \> \parbox[t]{5.25in}{You will need a classroom response device, or ``clicker'', for this course. These may be purchased in the GVSU bookstore and elsewhere. The preferred model is the Turning Point ResponseCard RF-LCD clicker, but other Turning Point devices may be used as well. 
}


\end{tabbing}

\section*{Overview of MTH 210}

\begin{tabbing}
\textbf{Catalog description}: \hspace{0.2in} \= 
\parbox[t]{5in}{A study of proof techniques used in mathematics. Intensive practice in reading mathematics, expository writing in mathematics, and constructing and writing mathematical proofs. Mathematical content includes elementary logic, congruence arithmetic, set theory, functions, equivalence relations, and equivalence classes. Offered fall and winter semesters.} 
\end{tabbing}
As the name of this course suggests, MTH 210 is all about communicating in mathematics. This takes on a few distinct but interrelated forms: \textbf{reading} mathematical writing with an analytical mind, \textbf{expressing} mathematical processes and results in \emph{written} and \emph{oral} form, and \textbf{listening} to others express themselves with a view towards mutual understanding. You communicate this way all the time in everyday life, but in mathematics these standard ways of interacting with the world take on some new dimensions. In particular, we need to learn some \emph{logic}, which is the language of mathematics; and our main goal in the course will be to become comfortable with the fundamental form of communication in mathematics: the \emph{mathematical proof}. Along the way you will also pick up some useful mathematical content such as set theory and number theory. 
\\ 

The basic outline of the course looks like this:
\begin{itemize}\itemsep1pt
	\item Fundamentals of logical statements and direct proofs. (Sundstrom Chapter 1)
	\item Staements, logical operators, logical equivalence, sets, and quantifiers. (Chapter 2)
	\item Direct proof, proof by contrapositive, proof by contradiction, proof by cases, and the division algorithm. (Chapter 3)
	\item Mathematical induction and recursion. (Chapter 4)
	\item Sets, operators on sets, set relationships, properties of set operations. (Chapter 5)
	\item Functions, injections, surjections, bijections, composition, and inverse functions. (Chapter 6)
	\item Relations, equivalence relations, and equivalence classes. (Chapter 7)
\end{itemize}

\textbf{Supplemental Writing Skills (SWS)}: MTH 210 is designated SWS (Supplemental Writing Skills) as described in the GVSU undergraduate catalog. The following statement comes from the catalog: ``SWS courses adhere to certain guidelines. Students turn in a total of at least 3000 words of writing. Part of that total may be essay exams, but a substantial amount of it is made up of finished essays, reports, or research papers. The instructor works with the students on revising drafts of papers, rather than simply grading the finished piece of writing. At least four hours of class time will be devoted to writing instruction. At least one third of the final grade in the course is based on the writing assignments.''

\section*{Learning objectives}

It's important that you have a clear idea of what you will need to know in this course. Here is a list of general objectives for students in MTH 210 that drives all the assessments, class activities, and graded work that we do. Aside from particular content competencies -- a list of which will be provided at the beginning of each section in an overview -- successful students in MTH 210 should be able to do the following. 
\begin{enumerate}
	\item Recall important facts about mathematical content and mathematical/logical reasoning and employ basic techniques of problem solving effectively. 
	\item Come to terms with a problem to prove by experimentation and conjecture.
	\item Construct well-formulated conjectures from empirical evidence. 
	\item Construct and write mathematical proofs using standard methods of mathematical proof including direct proofs, mathematical induction, case analysis, and counterexamples. 
	\item Read, summarize, and evaluate written mathematical proofs. 
	\item Communicate mathematical reasoning through clear writing, effective reading comprehension, and persuasive oral communication. 
	\item Become familiar with mathematical concepts outside of calculus and high school subject material. 
	\item Employ technology effectively both in exploring mathematical ideas and in communicating one's results. 
	\item Develop habits and attitudes attendant with effective problem-solving: good time management, positive approaches to work, acceptance of one's mistakes and others' feedback, graciousness in providing feedback to others, and a willingness to seek appropriate avenues of help when needed.
\end{enumerate}
It's important to realize that, perhaps unlike many math courses you've had in the past, MTH 210 is \emph{not} about getting the right answer to computational problems. You will actually not have very many computational exercises in this course. Instead, MTH 210 is about learning to think like a mathematician (even if you are not going to be a professional mathematician), developing sound problem-solving skills, and developing the ability to learn on your own and communicate your thought processes clearly. This tends to be a lot harder than getting the right answer to a computation. But we will go through the semester with the understanding that we are all works in progress and all committed to helping one another get better at this. If you put in the effort, keep a positive attitude, and commit to pursuing these goals, you WILL reach them. 


\section*{Workflow for MTH 210}

In a separate document called ``How MTH 210 Works'', I go into great detail about how a typical class week for MTH 210 is structured, how you should prepare for class, and what will go on in class. \textbf{Go read that document now if you haven't already.} A short summary of the day-to-day workflow of the course is as follows: 
\begin{itemize}[itemsep=0pt]
	\item Several days before a class meeting, you will be given an assignment called \emph{Guided Practice} for that meeting. It will consist of a list of \emph{Learning Objectives}, print and video resources for you to read and view, and a short list of activities and questions to respond to prior to class. Responses will be submitted electronically. 
	\item Upon arriving to class, you will take a short 1- or 2-question quiz over the Guided Practice using your clicker. Then we will engage in around 10 minutes of questions and answers based on the quiz results, questions from office hours and Piazza, and questions from the Guided Practice. 
	\item The balance of the class time (30--35 minutes usually) will be spent working on problems whose difficulty level goes beyond the basics. Some of these will be designated as individual problems with no collaboration allowed; other times you'll be given a choice of whether or not to collaborate with others, or you may be assigned to a group. The work on the in-class activities will usually be submitted at the end of the class session for a grade, although there may be an option to submit the work later if we run short on time. 
\end{itemize} 

To sum up, most of your time out of class will be focused on learning new concepts from print and video resources, preparing for class, working on your Proof Portfolio (see below), and asking questions in office hours and on Piazza. 

	
\section*{Assessment and grades}

Each piece of graded work in MTH 210 is carefully designed to measure your attainment or progress toward one or more of the objectives in the course. There are five basic kinds of graded work you will be doing: 

\begin{description}
	\item[Guided Practice (5\% of semester grade):] The Guided Practice activities that are submitted prior to class meetings will be graded on the basis of completeness, effort, and partial correctness. 
	\item[Class Preparation/Participation (5\%):] This category is a catch-all for preparation and activities other than Guided Practice and official in-class work. This includes the clicker quizzes at the beginning of classes, responses to reading questions that are posed on Piazza, occasional Piazza discussion assignments, and other activities assigned at my discretion. 
	\item[Class Work (20\%):] During class, you will work on problems that relate to, but extend, the basic concepts with which you became familiar through Guided Practice. These problems will sometimes be done individually, sometimes in groups, and sometimes with your choice of individual or groups. Typically, Class Work will be due at the end of class, but if we run short on time, I may give an extension to allow work to be completed outside of class. There will often be extra problems available for those who finish early; completion of extra problems will increase both your points earned and your points available.
	\item[Proof Portfolio (30\%):] You will complete ten mathematical problems that will be included in a portfolio. You will complete two preliminary drafts on each problem, receive feedback on each, and then assemble the final products in a printed copy at the end of the semester. The full set of specifications and guidelines for the portfolio are given in the document ``Proof Portfolio Information''. The problems themselves will be assigned on a rolling basis throughout the semester at a rate of about one problem per week. 
	\item[Midterm Exams (20\%):] There will be two exams during the semester, each counting for 10\% of your semester grade, to assess your mastery of the course material from the first and second thirds of the course. Dates for these are on the Course Calendar (go to \textbf{Blackboard}, then \textbf{Calendar}). 
	\item[Final exam (20\%):] A comprehensive final exam will be given on \textbf{Tuesday, December 11} from 12:00-1:50pm for Section 01, and \textbf{Wednesday, December 12} from 10:00--11:50am for Section 02. 
\end{description}



The method of assigning grades to your semester's body of work follows this table: 
\begin{center}
	\begin{tabular}{c|c||c|c}
	Percentage & Grade & Percentage & Grade \\ \hline
	93--100 & A & 73--76 & C \\
	90--92 & A- & 70--72 & C- \\
	87--89 & B+ & 67--69 & D+  \\
	83--86 & B & 60--66 & D\\
	80--82 & B- & 0--59 & F \\
	77--79 & C+ & 
	\end{tabular}
\end{center}
Note that no set distribution of grades is required for this course. Therefore it is entirely possible for the entire class to earn high grades if each student does excellent work in all phases of the course. 


\section*{Expectations for students, and the professor, in MTH 210}

Students in MTH 210 have the following expectations: 

\begin{description}
	\item[Preparation.] It is imperative that you prepare for class on a consistent basis. The Guided Practice will help you structure your time outside of class, but you also must learn to exercise good judgment, good time management, and personal sacrifice if you expect to do well. The university catalog states that a student should expect to spend 2--3 hours outside of class for every hour spent in class.  MTH 210 demands at least this much time from you, and probably more. Average successful students from past offerings of MTH 210 report that they spend \textbf{9--10 hours each week} doing course-related work. Therefore you should view your enrollment in MTH 210 as a commitment of at least 12 actual clock hours per week, three of which are spent together in class, and the other nine of which are spent outside of class under your management. You will handle this time commitment better if you spread it out over the course of an entire week in 1--2-hour blocks rather than a single ``cram'' session. Your time outside of class is to be focused on reading, watching tutorial videos, doing the Guided Practice for upcoming class sessions, working on Portfolio drafts, and asking questions in office hours and on Piazza. It is a major time commitment. Don't take it lightly, and ask for help early and often if you need it. 
	\item[Independence.] As explained in the ``How MTH 210 Works'' handout, you will be expected to develop independence as a learner. There will be little to no lecture done in class meetings. Instead, you'll be responsible for acquiring basic familiarity with new concepts through reading, videos, and Guided Practice and for asking questions on what you don't understand. This course can be difficult, and you will eventually get stuck on something; it is OK and expected. But don't expect to have anyone come to do your work for you! The initiative to learn and get better is yours. 
	\item[Academic honesty.] Please refer to the syllabus addendum on plagiarism and academic honesty, as well as the specific discussion in the directions for the portfolio project.  Note well that evidence of plagiarism in an assignment will result in a grade of zero and possible action under the guidelines of the GVSU catalog and student code.  You are expected to show integrity in all your work, and to encourage your peers to do likewise.  I reserve the right to discuss the nature of your work with you prior to assigning a grade on an assignment.   \emph{Note well: the proof portfolio is a completely independent exercise; no collaboration or use of external sources is permitted on it.} 
	\item[Respect.] You are expected to respect other people's work, feelings, space, and time. If you work in a group, you are expected to contribute fairly. In class, you are expected to be awake, attentive, and engaged. This means, in particular, it is not acceptable to use a laptop or smartphone to text, browse the web, view Facebook, or use the technology in ways that are distracting yourself or others from learning material in class. It is also not acceptable to sleep in class. Students who cannot abide by these basic guidelines may be asked to leave the class for the day, thereby forfeiting their Class Work and participation grades. 
	\item[Attendance.] This actually means two things. First, it means showing up for class. If you are running late, please email me to let me know; if you are late for an insufficient reason (for example, you overslept) then you may lose points on clicker quizzes and other activities that take place at the beginning of class. If you are absent for an insufficient reason, you may lose more points. The second thing ``attendance'' means is paying attention, especially outside of class. You are expected to view and contribute to discussion threads on Piazza and watch for announcements made there, on Blackboard, or in class. You are expected to check your email at least twice a day (once in the morning and once in the afternoon) for course updates. 	
\end{description}

On the flip side of this, each of you should expect the following from me, and hold me accountable if I fall short: 
\begin{itemize}
	\item I will work hard to design learning experiences for the class meetings that are interesting, meaningful, and lead students to learn something important. 
	\item I will be accessible to students, through posted office hours, appointments, and other means. 
	\item I will listen to student questions and give meaningful responses --- if not a direct answer, then an alternative line of questioning that leads the student to the right answer in a non-frustrating  way. 
	\item I will grade and return student work in a timely way. 
	\item I will carry out all grading and course policies fairly and in accordance with this syllabus unless extreme circumstances say otherwise, and then he will be flexible and fair to the rest of the class. 
	\item I will work hard to get to know you on a personal level, and I am committed to having fun in MTH 210 (and everything else). 
\end{itemize}
You will have multiple opportunities before the final course evaluations to let me know how I am doing on all of these fronts, and I always appreciate your feedback, formal or informal. 


\section*{For students with disabilities}

Grand Valley State University (GVSU) is committed to providing access to programs and facilities for all students, faculty, and staff. GVSU promotes the inclusion of individuals with disabilities as part of our commitment to creating a diverse, intercultural community. It is the policy of GVSU to comply with the Americans with Disabilities Act as amended by the ADA Amendment Act (2008), Section 504 of the Rehabilitation Act of 1973, and other applicable federal and state laws that prohibit discrimination on the basis of disability. GVSU will provide reasonable accommodations to qualified individuals with disabilities upon request. If there is any student in this class who has special needs because of learning, physical, or other disability, please contact me (Prof. Talbert) or the Disability Support Services office (200 STU, 616-331-2490). Furthermore, if you have a disability and think you will need assistance evacuating this classroom and/or building in an emergency situation, please make me aware so GVSU can develop a plan to assist you.

	
\section*{Disclaimer}
This syllabus is an outline of the policies and activities of this course. All information contained herein is subject to change. All changes will be announced to the entire class. 

\section*{Instructor Bio}

I'm an Associate Professor in the Mathematics Department at Grand Valley State University. Prior to coming to GVSU in 2011, I was Associate Professor of Mathematics and Computing Science and Director of the Engineering Program at Franklin College (2001--2011) and Assistant Professor of Mathematics at Bethel College, Indiana (1997--2001). I hold a Ph.D. in Mathematics from Vanderbilt University, where I was a Master Teaching Fellow at the Vanderbilt University Center for Teaching. My mathematical interests include cryptography and computational geometry. My interests in mathematics pedagogy include the use of technology to support active learning environments in the STEM (Science, Technology, Engineering, Mathematics) disciplines. I blog on these and other subjects for the C\emph{hronicle of Higher Education} at Casting Out Nines (\url{http:// chronicle.com/blognetwork/castingoutnines}).  I've been married to my wife, Cathy for 12 years, and our kids are Lucy (age 8), Penelope (age 6), and Harrison (age 3). We live in Allendale. My hobbies include reading, running, Indianapolis Colts football, and cooking. 




\end{document}