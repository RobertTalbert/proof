\documentclass[11pt]{article}

\pagestyle{empty}                       %no page numbers
\thispagestyle{empty}                   %removes first page number
\setlength{\parindent}{0in}               %no paragraph indents

\usepackage{enumitem}
\usepackage{fullpage}
\usepackage{palatino}
\usepackage{hyperref}

\begin{document}
	
\begin{center}
	\begin{Large}
		How MTH 210 Works
	\end{Large}
\end{center}
	
	
	One of the main goals of MTH 210 is to teach you to become \textbf{an independent learner who can learn new things continuously throughout the rest of your life}. This is a big deal. As you move through your degree and eventually into your career and your adult post-college life, your main value to the rest of the world and to the people you love is your ability to learn and grow without needing other people around to make it happen. There are many times in life where you MUST learn something, and you can't wait for the next semester at the local college to come around for you to sign up for a course. You have to take charge. You have to learn on your own. \\

	MTH 210 is in many ways the starting point for independent learning in the math program. Calculus was all fun and good, and you did well enough there to end up in MTH 210. But for the most part, Calculus is just another course where you are dependent on somebody for learning. The professor lectures, you take notes, you replicate the notes on tests, and you get a grade. But we're all smart enough to realize that the problems that the real world doesn't work like that. The problems that matter most to the world defy the lecture/example/homework/test/grade cycle. They require creative thinking, lots of failures and mistakes, careful analysis, and ultimately people who have the ability to adapt and learn on their own without depending on a professor to lecture to them first. So now that we're here,  we need to be realistic about the future and start preparing for a life of learning that matches up to the problems we'll need to solve. \\ 

	MTH 210 is \emph{not} set up along the traditional classroom model of lectures and examples in class, followed by homework outside of class. Instead, you will be expected to acquire basic familiarity with new concepts \emph{before} coming to class through a variety of means. This will effectively remove lectures (which don't work very well to begin with\footnote{\url{http://donaldclarkplanb.blogspot.com/2007/12/10-reasons-to-dump-lectures.html}}), despite what we may feel about them) from the classroom, freeing up massive amounts of time for us to deal with things that matter: answering questions, working on hard problems with each other, and hammering out an understanding of the material that cannot be conveyed by some person talking to you. \\

	By adopting this approach, you'll become less dependent on other people for your learning, and rediscover your ability to learn things on your own. (You have always had this ability, but it's easy to forget about.) And you'll be prepared to be a continuous learner, able to contribute something useful to the world. \\

	In MTH 210, here's how this will go. 
	\begin{itemize}[itemsep=0pt]
		\item Almost every class meeting is focused around a single topic, such as ``Proof by Contradiction'' or ``Inverse Functions''. To prepare for a class, you will be given an assignment called \textbf{Guided Practice}. Guided Practices consist of several parts: (1) A list of tasks, called \emph{Learning Objectives}, you should be able to perform reasonably before arriving at the class meeting; (2) a section of the Sundstrom textbook to which those objectives correspond, along with a reading assignment from that section; (3) a list of print and video resources that will help you attain the Learning Objectives; and (4) a short (\emph{short!}) list of simple activities to do and questions to respond to that will help you get to know the material and let you know how you are doing on the Learning Objectives. 
		\item Among the video resources you'll be given are videos from a playlist of YouTube videos that I am creating just for this course. There will eventually be over 100 videos, each around 3--5 minutes in length, that cover the entire breadth of the book. These are in some sense a replacement for the traditional classroom lectures you see in high school courses. These are much better than live lectures because you can watch them basically anywhere\footnote{If you have internet access issues, please see me. We might, for example, be able to burn some of these videos to a DVD or put them on a flash drive for offline access.}, pause them when you need a break, rewind when you need to hear something again, even watch them collaboratively online through a Google+ hangout. 
		\item You'll read the book, watch the videos, and do the Guided Practice activities before coming to class and submit those electronically. I'll be able to look at your work prior to class and make on-the-fly adjustments to my plans based on how well you're doing on the Guided Practice. 
		\item Once you come to class, the first thing we will do is have a very short quiz that covers one or two basic ideas from the Guided Practice. We'll take those quizzes using clickers, so you'll have immediate knowledge of your score and I'll have more data on how you're doing. 
		\item We'll follow the quizzes with 10 or so minutes of a question-and-answer session based on your quiz performance, your work on the Guided Practice, and any questions you may have submitted on the Piazza discussion boards prior to class. 
		\item The balance of the time we have left over (usually about 30--35 minutes) will be spent working on problems that invite you to go deeper into the material. This is our version of ``homework''. Instead of having 3--6 problems of homework each week to do outside of class, we will have 1--2 problems per meeting to work on inside of class, where you are free to collaborate, ask questions, screw up, etc. until you ``get it''. I'll have the advantage of checking in with every single one of you in the class in real time as you work, which allows me to tailor my instruction to you and prevents you from getting stuck on a problem for too long. Those in-class activities may be turned in at the end of class, but usually there will be a provision for finishing outside of class if we run out of time. 
	\end{itemize}

	You'll notice that most of the real work here is concentrated \emph{inside}, not outside, the classroom. In fact, there's really only three things you should feel responsible for doing outside of class: preparing for class meetings (which involves reading, watching videos, and working simple activities), working on your Proof Portfolio problems, and asking questions in office hours an on Piazza. All else, including what we usually call ``homework'', is done MWF 10-11 or 11-12. \\

	My hope is that this setup does two things. First, like I said above, I hope it helps you to become a more independent learner. Second, I hope that it helps make the workload of the course a little more bearable. MTH 210 is a traditionally hard course for a lot of people, primarily because of the workload. Hopefully the out-of-class arrangements here will help you focus that time on a couple of simple ongoing tasks (Guided Practice and asking questions) and on the hardest task in the course (Portfolio Problems) without a lot of other things competing for your mental space. \\

	This is a different setup than you are used to, and it will take a few class meetings to acclimate to the change. But I think you'll find that the effort will be worth it. And remember you can always voice your questions to me about what we are doing --- I will listen.
	
\end{document}


