\documentclass[addpoints]{exam}

% \pagestyle{empty}                       %no page numbers
% \thispagestyle{empty}                   %removes first page number
% \setlength{\parindent}{0in}               %no paragraph indents

\usepackage{fullpage}
\usepackage[tmargin = 0.5in, bmargin = 1in, hmargin = 1in]{geometry}     %1-inch margins
\geometry{letterpaper}                  
\usepackage{graphicx}
\usepackage{amssymb}

% Default packages
\usepackage{latexsym}
\usepackage{amsfonts}
\usepackage{amsmath}
\usepackage{amsthm}
\usepackage{palatino}
\usepackage{hyperref}
\usepackage{multicol}
\usepackage{multirow}
\usepackage{enumerate}
\usepackage{enumitem}


%% Definitions
\def\vi{\mathbf{i}}
\def\vj{\mathbf{j}}
\def\vk{\mathbf{k}}
\def\vr{\mathbf{r}}
\def\vu{\mathbf{u}}
\def\vv{\mathbf{v}}
\def\pageturn{\vfill
\begin{flushright}
	\begin{small}
		Continued $\rightarrow$
	\end{small}
\end{flushright}
\newpage}

\pagestyle{headandfoot}
\runningheadrule
\firstpageheader{\textbf{MTH 210 (Talbert)}}{\textbf{FINAL EXAM --- \numpoints \ points}}{\textbf{December 12, 2012}}
\runningheader{MTH 210}
{MTH 210 FINAL EXAM, Page \thepage\ of \numpages}
{Dec 12, 2012}
\firstpagefooter{}{}{}
\runningfooter{}{}{}

\begin{document}

		
\vspace*{0pt}

\noindent
Name: \underline{\hspace{2in}} \\


\noindent
\textbf{Instructions}: Welcome to your Final Exam. You may use three $3 \times 5$ notecards with notes and a calculator. You may NOT use any device that can communicate  with another device. The backs of each page are blank; use them if needed. On all questions other than multiple choice, give complete and correct solutions; answers without accompanying work will be given no credit. 

The test will end promptly at 12:00pm. No extensions or extra time will be given unless you have received prior permission from the instructor.

\begin{questions}

	
\uplevel{\emph{Items 1---12 are multiple choice questions that address a variety of learning objectives. Please circle the ONE response you believe is most correct. You do not need to justify your answer.}}

\question[2] Which of the following are statements? 
\begin{parts}
	\part $3^2 + 4^2 = 5^2$. 
	\part Prove or disprove that $3^2 + 4^2 = 5^2$. 
	\part $3^2 + 4^2 \neq 5^2$. 
	\part All of the above
	\part Just (a) and (c)
\end{parts}

\question[2] Suppose you wanted to prove that the following statement is FALSE: ``There exists a bijection $f: \mathbb{N} \to \mathbb{Z}$''. An appropriate technique for doing so would be
	\begin{parts}
		\part To give an example of a function $f: \mathbb{N} \to \mathbb{Z}$ that is a bijection
		\part To give an example of a function $f: \mathbb{N} \to \mathbb{Z}$ that is not a bijection
		\part To give a formal proof that all functions $f: \mathbb{N} \to \mathbb{Z}$ are bijections
		\part To give a formal proof that all functions $f: \mathbb{N} \to \mathbb{Z}$ fail to be bijections
		\part None of the above
	\end{parts}

\question[2] The negation of the statement ``If $A$, then $B$'' is
		\begin{parts}
			\part If $A$, then not $B$. 
			\part If not $A$, then $B$.
			\part Not $A$ and not $B$. 
			\part Not $A$ or not $B$. 
			\part None of the above
		\end{parts}

% \question[2] Consider the conditional statement: ``If $f$ is differentiable at $x=a$, then $f$ is continuous at $x=a$.'' Which of the following statements is logically equivalent to this statement? 
% 				\begin{parts}
% 					\part If $f$ is continuous at $x=a$, then $f$ is differentiable $x=a$. 
% 					\part If $f$ is not differentiable at $x=a$, then $f$ is not continuous at $x=a$ 
% 					\part If $f$ is not continuous at $x=a$, then $f$ is not differentiable at $x=a$. 
% 					\part All of the above
% 					\part Just (c) and (d)
% 				\end{parts}

\question[2] Under what conditions will the conditional statement $A \rightarrow (B \vee C)$ be true? 
	\begin{parts}
		\part When $A$ is true and $C$ is false
		\part When $A$ is false
		\part When $B$ is false
		\part All of the above
		\part Just (b) and (c)
	\end{parts}

\pageturn

\question[2] Consider the statement: ``For all integers $k$, $k^2 + 2k - 1$ is not a multiple of $4$.''. A mathematically correct strategy for proving this statement would be
	\begin{parts}
		\part Give an example of an integer $k$ such that $k^2 + 2k - 1$ is not a multiple of $4$. 
		\part Give an example of an integer $k$ such that $k^2 + 2k - 1$ \emph{is}  a multiple of $4$. 
		\part Assume $4$ divides $k^2 + 2k - 1$ and then provide an example of where this fails, thereby arriving at a contradiction. 
		\part Assume $4$ divides $k^2 + 2k - 1$ for some integer $k$ and then arrive at a contradiction. 
		\part Both (c) and (d)
	\end{parts}

% \question[2] 	What is the main difference between the Extended Principle of Mathematical Induction and the ``basic'' Principle of Mathematical Induction? 
% 	\begin{parts}
% 		\item The Extended Principle allows you to use induction on real numbers, not just natural numbers
% 		\item The Extended Principle allows you to start at a different base case 
% 		\item The Extended Principle changes the inductive hypothesis to assume the truth of each statement $P(1), P(2), \dots, P(k)$ for some $k \in \mathbb{N}$
% 		\item The Extended Principle has more words in it
% 	\end{parts}

\question[2] Suppose that $f$ and $g$ are two equal functions, the domain of $g$ is $\mathbb{Z}$, and that $g(-2) = 3$. Then based on this information alone, we can conclude that 
	\begin{parts}
		\item The domain of $f$ is $\mathbb{Z}$
		\item The codomain of $f$ is $\mathbb{Z}$
		\item $f(-2) = 3$
		\item All of the above
		\item Just (a) and (c)
	\end{parts}

% \question[2] To prove that three separate statements $A$, $B$, and $C$ are equivalent, we need to show
% \begin{parts}
% 	\item $A$ implies $B$
% 	\item $B$ implies $C$
% 	\item $C$ implies $A$
% 	\item All of the above
% 	\item Just (a) and (b)
% \end{parts}


\question[2] Which of the following functions $f: \mathbb{Z} \to \mathbb{Z}$ is a surjection? 
	\begin{parts}
		\part $f(a) = a^2$ 
		\part $f(a) = a^3$
		\part $f(a) = a \pmod {10}$
		\part All of the above
		\part None of the above
	\end{parts}


\question[2] Suppose $f: \mathbb{R} \rightarrow \mathbb{R}$ is given by $f(x) = 3 + 2^x + 4x$. This function is a bijection. The value of $f^{-1}(4)$ is 
	\begin{parts}
		\part $-4$
		\part $0$
		\part $1/35$
		\part $1/4$
		\part $35$
	\end{parts}

 	\pageturn



\question[2] Let $\sim$ be the relation on $\mathbb{R}$ given by $x \sim y$ if and only if $x-y$ is divisible by $5$. Then $\sim$ is 
	\begin{parts}
		\part Reflexive
		\part Symmetric
		\part Transitive
		\part All of the above
		\part Just (a) and (b) 
	\end{parts}

\question[2] Let $\sim$ be an equivalence relation on a nonempty set $A$ and suppose $a, b \in A$. Then $[a] = [b]$
	\begin{parts}
		\part Always
		\part Only if $a = b$
		\part Only if $a \sim b$
		\part Only if $[a]$ and $[b]$ are disjoint
		\part Never
	\end{parts}

\question[2] To prove that a function $f: A \to B$  is a surjection,
	\begin{parts}
		\part Let $a \in A$ and prove that $f(a)$ is a unique point in $B$. 
		\part Let $a, a' \in A$ with $a \neq a'$, and prove that $f(a) \neq  f(a')$. 
		\part Let $a, a' \in A$ with $f(a) \neq f(a')$, and prove $a \neq a'$.  
		\part Let $b \in B$, and prove there exists $a \in A$ such that $f(a) = b$. 
		\part Both (b) and (c)
	\end{parts}

\question[2] To show that a function $f: A \to B$  is \emph{not} an injection,
	\begin{parts}
		\part Show that there exists a point $a \in A$ such that $f(a)$ maps to two different points in $B$. 
		\part Show that there exists two points $a, a' \in A$ such that $f(a) \neq f(a')$.
		\part Show that there exists two points $a, a' \in A$ such that $a \neq a'$ but $f(a) = f(a')$. 
		\part Either (b) or (c)
		\part None of the above
	\end{parts}



\pageturn


%%% COMPUTATION

\question Below are a variety of computations to carry out. Do each one. You do not need to show work, but if you make a mistake, partial credit will be awarded only if there is supporting work. 
	\begin{parts}
		\part[6] Let $f: \mathbb{Z} \to \mathbb{Z}$ be defined by $f(a) = a^2 \pmod {11}$. Fill in the following table: 
		
		\begin{center}
			\begin{tabular}{c||c|c|c|c|c}
			$a$ & $-4$ & $-2$ & $0$ & $2$ & $4$ \\ \hline
			$f(a)$ & & & & & 
			\end{tabular}
		\end{center} 
		
		\vspace{1in}
		
		\part[6] Let $A = \{ 1, 2,3 , 4\}$, $B = \{ 3, 4, 5, 6\}$, and $C = \{ 6, 7\}$ Calculate the set $A \times (C - B)$. 
		
		\vspace{1in}
		
		\part[6] Find the integers $q$ and $r$ guaranteed by the Division Algorithm such that $1192009 = 6q + r$. 
		
		\vspace{1in}
		
		\part[6] Let $\sim$ be the relation on $\mathbb{Z}$ defined as follows: For all $a,b \in \mathbb{Z}$, $a \sim b$ if and only if $a-b$ is a multiple of 3. It can be proven that $\sim$ is an equivalence relation. Write the elements of $[1492]$ as a set in roster form. 
		
		
		
		\vspace{2in}
		
		
		\part[6] Let $\sim$ be the relation on the set $A = \{1, 2, 3, \dots, 10\}$ defined by $a \sim b$ if and only if $a$ divides $b$. Draw the directed graph for this relation. 
	\end{parts}

\pageturn
	
\question For each of the following statements, write the negation of the statement in symbolic form in which the negation symbol ($\neg$) is not used, and then write the negation as an English sentence. 
	\begin{parts}
		\part[6] $(\exists x \in \mathbb{R})(\cos(2x) = 2 \cos x)$
		
		\vspace{1.5in}
		
		
		\part[6] $(\forall x \in \mathbb{Z})((2|x) \Rightarrow (4|x))$ (The arrow ($\Rightarrow$) means ``implies''.) 
		
		\vspace{1.5in}
		
	\end{parts}
	
	

\question Consider the statement: 		\emph{If it is snowing and after 6:00am, then I will shovel my driveway.}

	\begin{parts}
		\part[6] Write a clear statement of the contrapositive of this sentence. 
		
		\vspace{2in}
		
		\part[6] Write a clear statement of the converse of this sentence. 
		
		% \vspace{1.5in}
		% 
		% \part[4] To which of the previous two statements (contrapositive and/or converse) is the original statement logically equivalent? Explain. 
	\end{parts}


\pageturn

% \uplevel{\emph{The next several items are problems to solve. Each item is tagged with a list of learning objectives it assesses. Be sure to give complete, clear, and correct solutions to each, not just answers.}}
% 	


% At least 1 of the following should be done on a setup level only.


\question Below are several proof-related tasks that we have undertaken repeatedly during the course. For each task, give a brief but complete outline of how a correct proof would be constructed in that situation. Be sure to state explicitly \textbf{what you would assume,  what you would try to prove, and a  specific strategy for how you would proceed from the beginning}. 
	\begin{parts}
		\part[8] Proving that two sets, $A$ and $B$, are equal via the ``choose an element'' approach

			\vspace{2.5in}
		
		\part[8] Proving the conditional statement ``If $P$, then $Q$'' by contraposition
		% 
		% 	\vspace{2in}
		% 
		% \part[8] Proving the conditional statement ``If $P$, then $Q$'' by contradiction
		% 
			\vspace{2.5in}
		
		\part[8] Proving that the predicate $P(n)$ is true for all natural numbers $n$ using mathematical induction
	\end{parts}

\pageturn 

\question[16] Choose EXACTLY ONE of the following and either prove the statement or disprove it. Circle the letter of the problem you are doing. 
	\begin{parts}
		\part For all integers $a,b$ and $d$ with $d \neq 0$, if $d$ divides $a$ or $d$ divides $b$, then $d$ divides $ab$. 
		\part For all functions $f: A \to B$ and $g: B \to C$, if $g \circ f: A \to C$ is an injection then  $f$ is an injection. 
		\part For each integer $n$, if $n$ is odd, then $8$ divides $n^2 - 1$. 
	\end{parts}


\pageturn

\question[16] Choose EXACTLY ONE of the following to prove. Circle the letter of the problem you are doing. 
	\begin{parts}
		\part Let $\sim$ be the relation on $\mathbb{Z}$ given by $a \sim b$ if and only if $a \equiv b \pmod 3$. For all natural numbers $n$, prove that $[10^n] = [1]$ under this relation. 
		\part Recall that the \emph{Fibonacci numbers} are the numbers $f_n$ defined by $f_1 = 1$, $f_2$, and then $f_k = f_{k-1} + f_{k-2}$ for all $k > 1$. Prove that for each natural number $n$, 
		\[ f_1^2 + f_2^2 + \cdots + f_n^2 = f_n f_{n+1} \]
		\part Prove that for every natural number $n$, 
		\[ \frac{1}{2} + \frac{1}{4} + \frac{1}{8} + \cdots + \frac{1}{2^n} = 1 - \frac{1}{2^n}\]
	\end{parts}

\pageturn


% \question Choose EXACTLY ONE of the following results about sets and prove it without using the ``algebra of sets'' approach or by referring to Theorems proven in the textbook. All sets listed are assumed to be subsets of some universal set $U$. 
% 	\begin{parts}
% 		\part $A - B = A \cap B^c$
% 		\part If $A \subseteq B$, then $A \cap B = B$. 
% 	\end{parts}

\question Consider the function $g: \mathbb{N} \times \mathbb{N} \to \mathbb{N}$ given by $g(a,b) = \gcd(a,b)$, the greatest common divisor of $a$ and $b$ (also known as greatest common factor). For example, $g(10, 8) = 2$ and $g(5,6) = 1$. 
	\begin{parts}
		\part[10] Prove or disprove: The function $g$ is injective. 
		
		\vspace{3in}
		
		\part[10] Prove or disprove: The function $g$ is surjective. 
		
		\vspace{2in}
		
	\end{parts}
		

\vfill


\question[10] The last problem on the exam consists of three short essay questions, given on a web form. The link for this is posted to Piazza already (thread \@177). Please submit that web form \textbf{by 8:00am on Thursday, December 13}. 

	
\end{questions}

		
\end{document}